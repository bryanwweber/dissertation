% arara: xelatex: { synctex: on, shell: off }
% arara: biber
% arara: xelatex: { synctex: on, shell: off }
% arara: sumatrapdf
\documentclass[12pt, letterpaper]{article}

%Set the package to import preambles
\usepackage{subfiles}

%Load graphicx here to specify options
\usepackage[final]{graphicx}

%Set the document font
\usepackage[no-math]{fontspec}
\setmainfont[Ligatures=TeX]{Times New Roman}
\setmonofont{Inconsolata}

%Set the text to double spacing
%According to hyperref README,
%setspace should be loaded first
\usepackage[doublespacing]{setspace}

%Set a command to easily skip a line
\newcommand{\blankline}{\vspace*{\baselineskip}}

%Set up biblatex
\usepackage[
    backend=biber,
    % url=false,
    doi=true,
    sorting=none,
    sortcites=true,
    maxbibnames=6,
    minbibnames=6,
    maxcitenames=2,
    mincitenames=1,
    citestyle=numeric-comp,
    firstinits=true,
    isbn=false
]{biblatex}
\addbibresource{C:/Users/\user/Documents/Github/dissertation/library.bib}

%Remove the "In:" from before the journal title for articles
\renewbibmacro{in:}{%
  \ifentrytype{article}{}{\printtext{\bibstring{in}\intitlepunct}}}

%Change the name of the bibliography section to "References"
\DefineBibliographyStrings{english}{bibliography = {References}}

%Set the sort order of the names in each bibliography entry
\DeclareNameAlias{default}{last-first}

%Don't print the article title. To print the title, add #1 to the last {}
\DeclareFieldFormat[article,incollection,unpublished]{title}{}

%Add "vol." and "no." before volume and issue.
\DeclareFieldFormat[article]{volume}{\bibstring{volume}\addspace #1}
\DeclareFieldFormat[article]{number}{\bibstring{number}\addspace #1}

%Ensure that a comma follows abbreviated journal titles.
\DeclareFieldFormat{journaltitle}{\mkbibemph{#1}\isdot}

%Put a comma between the volume and issue instead of period.
\renewbibmacro*{volume+number+eid}{%
  \printfield{volume}%
  \setunit{\addcomma\space}%<---- was \setunit*{\adddot}%
  \printfield{number}%
  \setunit{\addcomma\space}%
  \printfield{eid}}

%Add a comma after the journal title.
\renewbibmacro*{journal+issuetitle}{%
  \usebibmacro{journal}%
  \setunit*{\addcomma\addspace}%<---- was \setunit*{\addspace}%
  \iffieldundef{series}
    {}
    {\newunit
     \printfield{series}%
     \setunit{\addspace}}%
  \usebibmacro{volume+number+eid}%
  \setunit{\addspace}%
  \usebibmacro{issue+date}%
  \setunit{\addcolon\space}%
  \usebibmacro{issue}%
  \newunit}

%Only print URL if doi is not present.
\DeclareFieldFormat{url}{%
  \iffieldundef{doi}{%
    \mkbibacro{URL}\addcolon\space\url{#1}%
  }{%
  }%
}
\DeclareFieldFormat{urldate}{%
  \iffieldundef{doi}{%
    \mkbibparens{\bibstring{urlseen}\space#1}%
  }{%
  }%
}

%Remove publisher from being printed.
\renewbibmacro*{publisher+location+date}{%
  \printlist{location}%
  \setunit*{\addcomma\space}%
  \usebibmacro{date}%
  \newunit}

%Fix in-text full citations
\DeclareCiteCommand{\fullcite}
  {\usebibmacro{prenote}}
  {\usedriver
     {\defcounter{minnames}{99}%
      \defcounter{maxnames}{99}}
     {\thefield{entrytype}}}
  {\multicitedelim}
  {\usebibmacro{postnote}}

%Use fancy tables.
\usepackage{booktabs}

%Set up todo notes in the PDF file
\usepackage{todonotes}

%Use and set up the caption package for nicer captions.
\usepackage{caption}
\DeclareCaptionLabelFormat{bf}{\textbf{#1 #2}}
\captionsetup{
    font=small ,
    labelsep=colon ,
    labelformat=bf ,
    figurewithin=chapter ,
    tablewithin=chapter ,
}

\usepackage{titlesec}
\usepackage{titletoc}

\titleformat{\chapter}[display]{\normalfont\Huge\bfseries}{Chapter \thechapter}{0.7em}{}
\titleformat{\section}{\normalfont\LARGE\bfseries}{\thesection}{0.5em}{}
\titleformat{\subsection}{\normalfont\Large\bfseries}{\thesubsection}{1em}{}
\titleformat{\subsubsection}{\normalfont\large\bfseries}{\thesubsubsection}{1em}{}

\titlecontents{chapter}[0pc]{}{\bfseries Chapter \thecontentslabel\quad}{}{\titlerule*[0.5pc]{.}\contentspage}
\titlecontents{section}[1em]{}{\thecontentslabel\quad}{}{\titlerule*[0.5pc]{.}\contentspage}
\titlecontents{subsection}[2em]{}{\thecontentslabel\quad}{}{\titlerule*[0.5pc]{.}\contentspage}
\titlecontents{subsubsection}[3em]{}{\thecontentslabel\quad}{}{\titlerule*[0.5pc]{.}\contentspage}

\setcounter{secnumdepth}{3}
\setcounter{tocdepth}{3}

%Use the subfigure package
\usepackage{subfig}

%Various math improvements.
%Must be loaded before hyperref
\usepackage{mathtools}

%Set the math font. Has to come after mathtools because
%some font stuff gets overwritten.
\usepackage{unicode-math}
\unimathsetup{math-style=TeX}
\setmathfont[range=\mathup/{num}]{Times New Roman}
\setmathfont[range=\mathit/{greek,Greek,latin,Latin}]{Cambria Math}
\setmathfont[range=\mathup/{greek,Greek,latin,Latin}]{Cambria Math}
\setmathfont[range={"2212,"002B,"003D,"0028,"0029,"005B,"005D,"221A,
"2211,"2248,"222B,"007C,"2026,"2202,"00D7,"0302,"2261,"0025,"22C5,
"00B1,"2194,"21D4,"2260}]
{Cambria Math}

%Better looking fonts
\usepackage[final]{microtype}

%Allow table cells to span multiple rows.
\usepackage{multirow}

%Allow landscape rotated figures and captions.
\usepackage{afterpage}
\usepackage{rotating}
\usepackage{pdflscape}

%Set the root path where figures are stored.
\graphicspath{ {C:/Users/\user/Documents/Github/dissertation/figures/} }

%Set a convenience command for table cells that allow line breaks.
\newcommand{\linebreakcell}[2][c]{%
  \begin{tabular}[#1]{@{}c@{}}#2\end{tabular}}

%Use and set up the siunitx package for nice units printing.
\usepackage{siunitx}
\sisetup{%
    group-separator = {,},
    range-phrase = {\text{ to }},
    list-separator = {\text{, }},
    list-final-separator = {\text{, and }},
    list-pair-separator = {\text{ and }},
}%
\DeclareSIUnit\calorie{cal}
\DeclareSIUnit\atmosphere{atm}
\DeclareSIUnit\torr{torr}

%Declare convenience macros for printing the
%names of the alcohols.
\newcommand{\iPeOH}{\textit{i}-pentanol}
\newcommand{\nBuOH}{\textit{n}-butanol}
\newcommand{\sBuOH}{\textit{s}-butanol}
\newcommand{\tBuOH}{\textit{t}-butanol}
\newcommand{\iBuOH}{\textit{i}-butanol}

%The floatrow package allows multiple floats in a row
%and is set so that table captions are on top of the
%table.
\usepackage{floatrow}
\floatsetup[table]{style=plaintop}

%Use the titling package to allow easy access to custom title pages
\usepackage{titling}
\title{High Pressure Ignition Chemistry of Alternative Fuels}
\author{Bryan William Weber}

%Add bibliography and indices to the TOC
\usepackage{tocbibind}

%Improve handling of appendices
\usepackage{appendix}

%Use package that allows inline patching of commands. This is used in
%the appendices section.
\usepackage{xpatch}

%Use the bookmark package (which loads hyperref) so that only one
%compilation is necessary to get references.
\usepackage{bookmark}

%Set the color of the links and PDF metadata
\hypersetup{%
    pdfinfo={
        Title={High Pressure Ignition Chemistry of Alternative Fuels},
        Author={Bryan W. Weber},
    },
    colorlinks=true,
    citecolor=blue,
    linkcolor=black,
    plainpages=false,
    final,
}

%Allow lualatex to properly add links processed from pax files.
\usepackage{pdftexcmds}
\makeatletter
\let\pdfescapename=\pdf@escapename
\let\pdfstrcmp=\pdf@strcmp
\makeatother
\usepackage{pax}

%Allow to use \doi to link to DOI links.
\usepackage{doi}

%Allow inserting PDF documents directly to the output. According to
%http://tex.stackexchange.com/a/13660/32374, should come after hyperref
\usepackage{pdfpages}

%Do a better job with the automatic references. According to
%http://tex.stackexchange.com/a/1868/32374, should come after hyperref
\usepackage[capitalise, sort&compress]{cleveref}

%Set the auto-format names for the cleveref operations
\crefname{chapter}{Chapter}{Chapters}
\Crefname{chapter}{Chapter}{Chapters}
\crefname{section}{Sec.}{Secs.}
\Crefname{section}{Section}{Sections}
\crefname{subsection}{Sec.}{Secs.}
\Crefname{subsection}{Section}{Sections}
\crefname{subsubsection}{Sec.}{Secs.}
\Crefname{subsubsection}{Section}{Sections}
\crefname{figure}{Fig.}{Figs.}
\Crefname{figure}{Figure}{Figures}
\crefname{table}{Table}{Tables}
\Crefname{table}{Table}{Tables}
\crefname{equation}{Eq.}{Eqs.}
\Crefname{equation}{Equation}{Equations}
\crefname{appchap}{Appendix}{Appendices}
\Crefname{appchap}{Appendix}{Appendices}

\newcommand{\creflastconjunction}{, and~}
\newcommand{\crefrangeconjunction}{--}

%Set the size of the margins and the paper
%According to http://tex.stackexchange.com/a/26592/32374
%this should go after hyperref
\usepackage[margin=1in, letterpaper]{geometry}

%Set up the page numbers
%This has to go after geometry so the page number is centered
\usepackage{fancyhdr}
\pagestyle{fancy}
\fancyhf{}
\fancyfoot[C]{\thepage}
\renewcommand{\headrulewidth}{0pt}


\begin{document}
\section{Introduction}
\label{sec:intro-introduction}

The world relies heavily on combustion to provide energy in useful forms for
human consumption; combustion currently represents over 80\% of the world
energy production \cite{Sims2007} and is predicted to decrease in importance
only slightly by 2040 \cite{EIA2013}. In particular, the transportation sector
accounts for nearly 40\% of the energy use in the United States and of that,
more than 90\% is supplied by combustion of fossil fuels \cite{MER2013}.
Unfortunately, emissions from the combustion of traditional fossil fuels have
been implicated in a host of deleterious effects on human health and the
environment \cite{Avakian2002} and fluctuations in the price of traditional
fuels can have a negative impact on the economy \cite{Owen2010}.

Despite its shortcomings, combustion is currently the only energy conversion
mechanism that offers the immediate capability to generate the sheer amount
of energy required to run the modern world. Since we cannot eliminate
combustion as an important energy conversion method, we must instead ameliorate
the shortcomings of a primarily combustion-based energy economy. A two-pronged
approach has developed to achieve the necessary improvements. These prongs
include: 1) development of new fuel sources and 2) development of new
combustion technologies. First, using new sources of fuel for combustion-based
energy conversion can reduce the economic impact of swings in the price of
current fuels, in addition to potentially reducing emissions. Second, using new
combustion technologies can reduce harmful emissions while simultaneously
increasing the efficiency of combustion processes, thereby reducing fuel
consumption.

Many new sources of fuels have been investigated recently. The most promising
of these in the long term are renewable biological sources, which are used to
produce fuels known as biofuels. The advantage of biofuels over traditional
fuels lies in their feedstocks. Whereas traditional fuel feedstocks generally
require millions of years to be produced, biofuel feedstocks are replenished
on an annual basis. Furthermore, biofuels offer the potential to offset carbon
emissions created from their combustion by reusing the emitted carbon to grow
the plants from which the fuels are produced. However, the combustion
properties of biofuels may be substantially different from the traditional
fuels they are intended to replace. This makes it difficult to quickly switch
the energy economy to biofuels and necessitates medium-term investigation of
alternative sources for traditional fuels. These sources include shale oil and
liquefied coal, which have different chemical compositions than traditional
fuel sources and therefore fuels made from these alternative sources have
different combustion properties. Collectively, all of these fuels created from
non-traditional sources are known as alternative fuels.

In addition to new fuel sources, new engine technologies are rapidly being
developed. These include engines capable of operating in favorable combustion
regimes, such as so-called Low Temperature Combustion (LTC) engines and
Homogeneous Charge Compression Ignition (HCCI) engines. These devices avoid
regions in the temperature-equivalence ratio space where combustion generates a
large amount of emissions and attempt to operate in regions where efficiency is
maximized and emissions are reduced. Other devices, such as the well-known
catalytic converter, operate on the exhaust after it leaves the cylinder to
improve emissions characteristics.

Neither of these approaches is able to mitigate all of the negative impacts of
combustion by itself. By switching to biofuels but retaining the same engines,
the efficiency and emissions targets may not be met; by only developing new
engines, our sources of fuel will continue to cause economic distress, turmoil,
and negative effects on the environment. It will take a concerted effort to
bring these two pathways of innovation together.

Unfortunately, there are many roadblocks on the way to combining new fuels in
new engines. For instance, one can imagine the design and testing process of
new engines and fuels becoming circular: the ``best'' alternative fuel should be
tested in the ``best'' engine, but the ``best'' engine depends on which is selected
as the ``best'' alternative fuel. One way to cut this circle short is by
employing computer-aided design and modeling of new engines with new fuels to
design engines to be fuel-flexible. Accurate and predictive models of
combustion processes can be used to computationally test the efficacy of new
technologies and fuels before they undergo expensive real-world testing. The
key to this process is the development of accurate and predictive combustion
models.

Substantial work has been put forth recently to develop and validate predictive
combustion models for several alternative fuels. These studies include
calculation and measurement of reaction rate coefficients, measurement of
global and local combustion properties, and development of model construction
methodologies. Nevertheless, much of the work is still ongoing, and there is
substantial room for extending the state-of-the-art knowledge, especially at
high-pressure conditions relevant to combustion in engines.

Chemical kinetic models for the combustion of large molecules are typically
built in a hierarchical fashion, as described by \textcite{Westbrook1984}. That
is, the model for the combustion of heptane contains the model for the
combustion of hexane added to the model of combustion for pentane, and so on
down to the models for hydrogen and carbon monoxide combustion. Therefore, it
is important to thoroughly validate the models for smaller species while
building models of higher hydrocarbons and other molecular types. Work has been
ongoing to explore the chemistry of small molecules for decades. Notable recent
kinetic mechanisms to emerge from this work include the GRI-Mech series of
mechanisms (most recently, version 3.0 \cite{Smith}), USC-Mech v2
\cite{Wang2007}, and the AramcoMech series of mechanisms, most recently version
1.3 \cite{Metcalfe2013}.

\end{document}