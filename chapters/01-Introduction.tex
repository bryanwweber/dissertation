% arara: xelatex: { synctex: on, shell: off }
% arara: biber
% arara: xelatex: { synctex: on, shell: off }
% arara: sumatrapdf
\documentclass[12pt, letterpaper]{article}

%Set the document font
\usepackage{fontspec}
\setmainfont[Renderer=Basic,Ligatures=TeX]{Times New Roman}

%Set the size of the margins and the paper
\usepackage[margin=1in, letterpaper]{geometry}

%Set the color of the links and PDF metadata
\usepackage[
    colorlinks=true,
    citecolor=blue,
    linkcolor=black
]{hyperref}

\hypersetup{%
    pdfinfo={
        Title={High Pressure Ignition Chemistry of Alternative Fuels},
        Author={Bryan W. Weber}
    }
}
%Set up the page numbers
%This has to go after geometry so the page number is centered
\usepackage{fancyhdr}
\pagestyle{fancy}
\fancyhf{}
\fancyfoot[C]{\thepage}
\renewcommand{\headrulewidth}{0pt}

%Set a command to easily skip a line
\newcommand{\blankline}{\vspace*{\baselineskip}}

%Set up biblatex
\usepackage[
    backend=biber,
    url=false,
    doi=true,
    sorting=none,
    sortcites=true,
    maxbibnames=6,
    minbibnames=6,
    maxcitenames=2,
    mincitenames=1,
    citestyle=numeric-comp,
    firstinits=true,
    isbn=false
]{biblatex}
\addbibresource{../library.bib}

%Remove the "In:" from before the journal title for articles
\renewbibmacro{in:}{%
  \ifentrytype{article}{}{\printtext{\bibstring{in}\intitlepunct}}}

%Set the sort order of the names in each bibliography entry
\DeclareNameAlias{default}{last-first}

%Don't print the article title. To print the title, add #1 to the last {}
\DeclareFieldFormat[article,incollection,unpublished]{title}{}

%Add Vol. and No. before volume and issue.
\DeclareFieldFormat[article]{volume}{\bibstring{volume}\addspace #1}
\DeclareFieldFormat[article]{number}{\bibstring{number}\addspace #1}

%Put a comma between the volume and issue instead of period
\renewbibmacro*{volume+number+eid}{%
  \printfield{volume}%
  \setunit{\addcomma\space}%<---- was \setunit*{\adddot}%
  \printfield{number}%
  \setunit{\addcomma\space}%
  \printfield{eid}}

%Add a comma after the journal title
\renewbibmacro*{journal+issuetitle}{%
  \usebibmacro{journal}%
  \setunit*{\addcomma\addspace}%
  \iffieldundef{series}
    {}
    {\newunit
     \printfield{series}%
     \setunit{\addspace}}%
  \usebibmacro{volume+number+eid}%
  \setunit{\addspace}%
  \usebibmacro{issue+date}%
  \setunit{\addcolon\space}%
  \usebibmacro{issue}%
  \newunit}

%Set the text to double spacing
\usepackage[doublespacing]{setspace}

%Packages not present in main.tex preamble
\usepackage{booktabs}

\def\chapterautorefname~#1\null{Chap.~#1\null}
\def\sectionautorefname~#1\null{Sec.~#1\null}
\def\subsectionautorefname~#1\null{Sec.~#1\null}
\def\figureautorefname~#1\null{Fig.~#1\null}
\def\tableautorefname~#1\null{Table~#1\null}
\def\equationautorefname~#1\null{Eq.~(#1)\null}

\newcommand{\Autoref}[1]{%
  \begingroup%
  \def\chapterautorefname~##1\null{Chapter~##1\null}%
  \def\sectionautorefname~##1\null{Section~##1\null}%
  \def\subsectionautorefname~##1\null{Sub--Section~##1\null}%
  \def\figureautorefname~##1\null{Figure~##1\null}%
  \def\tableautorefname~##1\null{Table~##1\null}%
  \def\equationautorefname~##1\null{Equation~##1\null}%
  \autoref{#1}%
  \endgroup%
}

\usepackage[font={footnotesize}]{caption}

\usepackage{mathtools}

\usepackage{multirow}

\graphicspath{ {../figures/} }

\newcommand{\linebreakcell}[2][c]{%
  \begin{tabular}[#1]{@{}c@{}}#2\end{tabular}}

%End of extra imports

\begin{document}
\section{Introduction}
\label{sec:intro-introduction}

The world relies heavily on combustion to provide energy in useful forms for 
human consumption; combustion currently represents over 80\% of the world 
energy production \cite{Sims2007} and is predicted to decrease in importance 
only slightly by 2040 \cite{EIA2013}. In particular, the transportation sector 
accounts for nearly 40\% of the energy use in the United States and of that, 
more than 90\% is supplied by combustion of fossil fuels \cite{MER2013}. 
Unfortunately, emissions from the combustion of traditional fossil fuels have 
been implicated in a host of deleterious effects on human health and the 
environment \cite{Avakian2002} and fluctuations in the price of traditional 
fuels can have a negative impact on the economy \cite{Owen2010}.

Despite its shortcomings, combustion is currently the only energy conversion 
mechanism that offers the immediate capability to generate the sheer amount 
of energy required to run the modern world. Since we cannot eliminate 
combustion as an important energy conversion method, we must instead ameliorate 
the shortcomings of a primarily combustion-based energy economy. A two-pronged 
approach has developed to achieve the necessary improvements. These prongs 
include: 1) development of new fuel sources and 2) development of new 
combustion technologies. First, using new sources of fuel for combustion-based 
energy conversion can reduce the economic impact of swings in the price of 
current fuels, in addition to potentially reducing emissions. Second, using new 
combustion technologies can reduce harmful emissions while simultaneously 
increasing the efficiency of combustion processes, thereby reducing fuel 
consumption.

Many new sources of fuels have been investigated recently. The most promising 
of these in the long term are renewable biological sources, which are used to 
produce fuels known as biofuels. The advantage of biofuels over traditional 
fuels lies in their feedstocks. Whereas traditional fuel feedstocks generally 
require millions of years to be produced, biofuel feedstocks are replenished 
on an annual basis. Furthermore, biofuels offer the potential to offset carbon 
emissions created from their combustion by reusing the emitted carbon to grow 
the plants from which the fuels are produced. However, the combustion 
properties of biofuels may be substantially different from the traditional 
fuels they are intended to replace. This makes it difficult to quickly switch 
the energy economy to biofuels and necessitates medium-term investigation of 
alternative sources for traditional fuels. These sources include shale oil and 
liquefied coal, which have different chemical compositions than traditional 
fuel sources and therefore fuels made from these alternative sources have 
different combustion properties. Collectively, all of these fuels created from 
non-traditional sources are known as alternative fuels. 

In addition to new fuel sources, new engine technologies are rapidly being 
developed. These include engines capable of operating in favorable combustion 
regimes, such as so-called Low Temperature Combustion (LTC) engines and 
Homogeneous Charge Compression Ignition (HCCI) engines. These devices avoid 
regions in the temperature-equivalence ratio space where combustion generates a 
large amount of emissions and attempt to operate in regions where efficiency is 
maximized and emissions are reduced. Other devices, such as the well-known 
catalytic converter, operate on the exhaust after it leaves the cylinder to 
improve emissions characteristics.

Neither of these approaches is able to mitigate all of the negative impacts of 
combustion by itself. By switching to biofuels but retaining the same engines, 
the efficiency and emissions targets may not be met; by only developing new 
engines, our sources of fuel will continue to cause economic distress, turmoil, 
and negative effects on the environment. It will take a concerted effort to 
bring these two pathways of innovation together.

Unfortunately, there are many roadblocks on the way to combining new fuels in 
new engines. For instance, one can imagine the design and testing process of 
new engines and fuels becoming circular: the “best” alternative fuel should be 
tested in the “best” engine, but the “best” engine depends on which is selected 
as the “best” alternative fuel. One way to cut this circle short is by 
employing computer-aided design and modeling of new engines with new fuels to 
design engines to be fuel-flexible. Accurate and predictive models of 
combustion processes can be used to computationally test the efficacy of new 
technologies and fuels before they undergo expensive real-world testing. The 
key to this process is the development of accurate and predictive combustion 
models.

Substantial work has been put forth recently to develop and validate predictive 
combustion models for several alternative fuels. These studies include 
calculation and measurement of reaction rate coefficients, measurement of 
global and local combustion properties, and development of model construction 
methodologies. Nevertheless, much of the work is still ongoing, and there is 
substantial room for extending the state-of-the-art knowledge, especially at 
high-pressure conditions relevant to combustion in engines.

Chemical kinetic models for the combustion of large molecules are typically 
built in a hierarchical fashion, as described by \textcite{Westbrook1984}. That 
is, the model for the combustion of heptane contains the model for the 
combustion of hexane added to the model of combustion for pentane, and so on 
down to the models for hydrogen and carbon monoxide combustion. Therefore, it 
is important to thoroughly validate the models for smaller species while 
building models of higher hydrocarbons and other molecular types. Work has been 
ongoing to explore the chemistry of small molecules for decades. Notable recent 
kinetic mechanisms to emerge from this work include the GRI-Mech series of 
mechanisms (most recently, version 3.0 \cite{Smith}), USC-Mech v2 
\cite{Wang2007}, and the AramcoMech series of mechanisms, most recently version 
1.3 \cite{Metcalfe2013}.

\end{document}