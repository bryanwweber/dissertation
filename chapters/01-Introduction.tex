% arara: lualatex: { synctex: on, shell: off }
% arara: biber
% arara: lualatex: { synctex: on, shell: off }
% arara: sumatrapdf
\documentclass[../main.tex]{subfiles}

\begin{document}
\label{sec:intro-introduction}

\section{Background}

The world relies heavily on combustion to provide energy in useful forms for
human consumption; combustion currently represents over \SI{80}{\percent} of the world
energy production \cite{Sims2007} and is predicted to decrease in importance
only slightly by 2040 \cite{EIA2013}. In particular, the transportation sector
accounts for nearly \SI{40}{\percent} of the energy use in the United States and of that,
more than \SI{90}{\percent} is supplied by combustion of fossil fuels \cite{MER2013}.
Unfortunately, emissions from the combustion of traditional fossil fuels have
been implicated in a host of deleterious effects on human health and the
environment \cite{Avakian2002} and fluctuations in the price of traditional
fuels can have a negative impact on the economy \cite{Owen2010}.

Despite its shortcomings, combustion is currently the only energy conversion
mechanism that offers the immediate capability to generate the sheer amount
of energy required to run the modern world. Since we cannot eliminate
combustion as an important energy conversion method, we must instead ameliorate
the shortcomings of a primarily combustion-based energy economy. A two-pronged
approach has been developed to achieve the necessary improvements. These approaches
include: 1) development of new fuel sources and 2) development of new
combustion technologies. First, using new sources of fuel for combustion-based
energy conversion can reduce the economic impact of swings in the price of
current fuels, in addition to potentially reducing emissions. Second, using new
combustion technologies can reduce harmful emissions while simultaneously
increasing the efficiency of combustion processes, thereby reducing fuel
consumption.

Many new sources of fuels have been investigated recently. The most promising
of these in the long term are renewable biological sources, which are used to
produce fuels known as biofuels. The advantage of biofuels over traditional
fuels lies in their feedstocks. Whereas traditional fuel feedstocks generally
require millions of years to be produced, biofuel feedstocks are replenished
on an annual basis. Furthermore, biofuels offer the potential to offset carbon
emissions created from their combustion by reusing the emitted carbon to grow
the plants from which the fuels are produced.

However, the combustion
properties of biofuels may be substantially different from the traditional
fuels they are intended to replace. This makes it difficult to quickly switch
the energy economy to biofuels and necessitates medium-term investigation of
alternative sources for traditional fuels. These sources include shale oil and
liquefied coal, which have different chemical compositions than traditional
fuel sources and therefore fuels made from these alternative sources have
different combustion properties. Collectively, all of these fuels created from
non-traditional sources are known as alternative fuels.

In addition to new fuel sources, new engine technologies are rapidly being
developed. These include engines capable of operating in favorable combustion
regimes, such as so-called Low Temperature Combustion (LTC) engines and
Homogeneous Charge Compression Ignition (HCCI) engines. These devices avoid
regions in the temperature-equivalence ratio space where combustion generates a
large amount of emissions and operate in regions where efficiency is
maximized and emissions are reduced. Other devices, such as the well-known
catalytic converter, operate on the exhaust after it leaves the cylinder to
improve emissions characteristics.

Neither of these approaches is able to mitigate all of the negative impacts of
combustion by itself. By switching to biofuels but retaining the same engines,
the efficiency and emissions targets may not be met; by only developing new
engines, our sources of fuel will continue to cause economic distress, turmoil,
and negative effects on the environment. It will take a concerted effort to
bring these two pathways of innovation together.

Unfortunately, there are many roadblocks on the way to combining new fuels in
new engines. For instance, one can imagine the design and testing process of
new engines and fuels becoming circular: the ``best'' alternative fuel should be
tested in the ``best'' engine, but the ``best'' engine depends on which is selected
as the ``best'' alternative fuel. One way to cut this circle short is by
employing computer-aided design and modeling of new engines with new fuels to
design engines to be fuel-flexible. Accurate and predictive models of
combustion processes can be used to computationally test the efficacy of new
technologies and fuels before they undergo expensive real-world testing. The
key to this process is the development of accurate and predictive combustion
models.

Substantial work has been put forth recently to develop and validate predictive
combustion models for several alternative fuels. These studies include
calculation and measurement of reaction rate coefficients, measurement of
global and local combustion properties, and development of model construction
methodologies. Nevertheless, much of the work is still ongoing, and there is
substantial room for extending the state-of-the-art knowledge, especially at
high-pressure conditions relevant to combustion in engines.

A combustion model for the combustion of a given fuel in a given device
must necessarily accurately model the complete interaction of the operating
elements of the device. This includes sub-models for the chemical reactions that the
fuel and oxidizer undergo as well as the interaction of the fuel/oxidizer
mixture with the operation of the device. The first of these is known as
a chemical kinetic model or reaction mechanism; the second typically includes
such effects as turbulence interaction, heat transfer, liquid spray
dynamics, fluid mechanics, etc., each of which are typically modeled independently.

Chemical kinetic models for the combustion of large molecules are typically
built in a hierarchical fashion, as described by \textcite{Westbrook1984}. That
is, the model for the combustion of heptane contains the model for the
combustion of hexane added to the model of combustion for pentane, and so on
down to the models for hydrogen and carbon monoxide combustion. Therefore, it
is important to thoroughly validate the models for smaller species while
building models of higher hydrocarbons and other molecular types. Work has been
ongoing to explore the chemistry of small molecules for decades. Notable recent
kinetic mechanisms to emerge from this work include the GRI-Mech series of
mechanisms (most recently, version 3.0 \cite{Smith}), USC-Mech v2
\cite{Wang2007}, and the AramcoMech series of mechanisms, most recently version
1.3 \cite{Metcalfe2013}.

Validation of kinetic models for the combustion of larger fuels
has proceeded in parallel with the small molecule chemistry. Given their projected
importance to combustion, one focus of the larger molecule work has naturally
been on biofuels. These biofuels typically include chemical species such as
alcohols and esters – neat alcohols can be used as fuels, while esters are
typically found as components of biodiesel fuels. A review by \textcite{Kohse-Hoinghaus2010}
covers much of the experimental data available until 2010. Since then an
enormous amount of data has been produced for both alcohols and esters.
Since the focus of this study is on alcohols, I will highlight alcohols
in the following sections.

Model construction and validation has also been focused on alternative ``traditional''
fuels, that is, fuels that are chemically similar to traditional fuels but
produced from alternative sources such as shale oil or coal liquefaction.
Traditional fuels and alternative ``traditional'' fuels typically contain
hundreds or thousands of chemical components. This makes building and using
models containing every species present in the fuel intractable on current
computer hardware. Therefore, a more useful approach to building models for
these fuels is to define a surrogate fuel. Surrogate fuels are made of a
limited number of chemical components to ensure that model building and use
are tractable, but the components are chosen so that the surrogate fuel
faithfully reproduces the physical and chemical properties of the real fuel.

Much progress has been made recently to construct surrogates for typical
transportation fuels. For instance, work on diesel surrogate formulation has
recently been reviewed by \textcite{Pitz2011}, work on gas turbine fuel
surrogates has been briefly summarized by \textcite{Dooley2012}, and work on
gasoline surrogates has been summarized in the work of \textcite{Anand2011,Pitz2007}.
One typical component class in the surrogate formulations is a cycloalkane or
alkyl-cycloalkane (collectively known as naphthenes), due to this class' presence
in nearly all transportation fuels \cite{Pitz2007, Briker2001, Farrell2007,
Edwards2007}. One particular cycloalkane, methylcyclohexane (MCH),
has been suggested in several surrogate formulations, including those by
\textcite{Bieleveld2009,Naik2005}. Recent work on MCH combustion will also
be highlighted in the following sections.

\section{Recent Work on the Combustion of Alcohols}

Among the alcohols being considered as biofuels, two criteria are typically
used to judge the suitability of a species: 1) its ease of production and
2) its potential as a ``drop-in'' replacement for current fuels. Because
of these criteria (among others), much research recently has focused on
the isomers of butanol, the C$_4$ alcohols, and \iPeOH{}, a C$_5$
alcohol. This is because these fuels are easy to produce by a number of biological
pathways \cite{Peralta-Yahya2012} and offer similar properties as
gasoline for use in typical automotive transportation applications \cite{Afeefy2014,
Davis2013}.

One of the most common biofuels currently in use is ethanol (C$_2$H$_6$O).
Although ethanol is ubiquitous at gasoline pumps, it suffers from several
disadvantages that suggest it needs to be replaced \cite{Niven2005}. In particular, ethanol
has a much lower energy density than gasoline, reducing volumetric fuel
economy, and ethanol is typically produced from crops that would otherwise
be used as food sources \cite{Somma2009}.

\textit{n}-Butanol has recently been identified as one of a suite of so
called ``second generation'' biofuels intended to supplement or replace
the ``first generation'' biofuels currently in use, such as ethanol \cite{Harvey2011,Nigam2011}.
The second generation biofuels will help alleviate some of the problems
identified with the first generation biofuels, including concerns about
feedstocks. In addition to the normal (\textit{n}) isomer, there are three
other isomers of butanol--- \textit{s}-, \textit{i}-, and \tBuOH{}.
Biological production pathways have been identified for \textit{n}-, \textit{s}-,
and \iBuOH{} \cite{Nigam2011,Smith2010}, but \tBuOH{} is a petroleum derived product.
Nevertheless, \tBuOH{} is currently used as an octane enhancer in gasoline.

In the last five years, research into the combustion characteristics of
the isomers of butanol has exploded, so exemplary references are
provided here except for the articles of particular interest to this work.
In addition to applied engine research \cite{Dernotte2009, Szwaja2010, Kim2011},
fundamental combustion measurements have been made using many different
systems. These include laminar flame speeds \cite{Veloo2011a}, jet-stirred reactor
chemistry \cite{Dagaut2009}, low-pressure flame structure \cite{Hansen2011a,Hansen2013},
atmospheric pressure flame structure \cite{Grana2010}, pyrolysis \cite{VanGeem2010, Cai2012, Cai2013},
flow reactors \cite{Lefkowitz2012,Heyne2013}, and ignition delays, which will be discussed
in more detail shortly. Other researchers have measured or calculated the reaction
rate constants of reactions of butanol with various radicals, including OH \cite{Stranic2013a,
Pang2012, Pang2012a, Seal2013, Pang2012b, El-Nahas2012, Zhou2011, Moc2010, Vasu2010}, HO$_2$
\cite{Zhou2012, Alecu2012, Black2010a}, and CH$_3$ \cite{Katsikadakos2013, Katsikadakos2012}.

Several studies of ignition delay of the butanol isomers have been conducted
in both STs (STs) and Rapid Compression Machines (RCMs), including work in STs by \textcite{Moss2008,
Black2010, Noorani2010, Zhang2012, Stranic2012, Yasunaga2012, Heufer2011,
Vranckx2011, Zhu2014} and work in RCMs by \textcite{Weber2011,
Weber2013, Weber2013a, Karwat2011a}. These studies have covered a wide range of
temperature-pressure regimes, from \SIrange{1}{90}{\bar} and \SIrange{675}{1800}{\kelvin}.

Among the ST ignition studies, \textcite{Moss2008} have done
measurements for all four isomers of butanol at \SIlist{1;4}{\bar} and
\SIrange{1200}{1800}{\kelvin}, over equivalence ratios of $\phi = \numlist{0.5;1.0;2.0}$
and fuel mole percentages of \SIlist{0.25;0.5;1.0}{\percent}.
\textcite{Black2010} investigated autoignition for \nBuOH{} from
\SIrange{1100}{1800}{\kelvin} and \SIlist{1;2.6;8}{\atmosphere} over equivalence ratios of
$\phi = \numlist{0.5;1.0;2.0}$ and fuel mole percentages of \SIlist{0.6;0.75;3.5}{\percent}.
\textcite{Noorani2010} investigated ignition of the primary alcohols
from C$_1$--C$_4$ (methanol to butanol) at pressures of \SIlist{2;10;12}{\atmosphere} under dilute
conditions for equivalence ratios $\phi = \numlist{0.5;1.0;2.0}$, and temperatures
from \SIrange{1070}{1760}{\kelvin}. \textcite{Zhang2012} measured ignition
delays of \nBuOH{} at pressures of \SIlist{2;10}{\atmosphere},
temperatures in the range of \SIrange{1200}{1650}{\kelvin}, and for equivalence
ratios of \numlist{0,5;1.0;2.0}. \textcite{Stranic2012} measured ignition
delays of all four isomers of butanol over the pressure range \SIrange{1.5}{43}{\atmosphere},
temperature range \SIrange{1050}{1600}{\kelvin}, and equivalence ratios of
\numlist{0.5;1.0}. These studies showed generally good agreement of
the ignition delays for \nBuOH{}, but \textcite{Stranic2012}
found that their ignition delays for the other isomers of butanol were
shorter than the ignition delays measured by \textcite{Moss2008}.
\textcite{Stranic2012} were unable to determine the reason for the
discrepancy.

\textcite{Yasunaga2012} measured ignition delays of \textit{s}-,
\textit{t}-, and \iBuOH{} at a pressure of \SI{3.5}{\atmosphere}
and temperatures between \SIrange{1250}{1800}{\kelvin}. In addition,
\textcite{Yasunaga2012} measured reactant, intermediate, and product
species during pyrolysis of all four butanol isomers by sample extraction
from their ST and analysis by gas chromatography. Other researchers
have measured species profiles during the pyrolysis of butanol isomers
in a ST by optical techniques, including \textcite{Cook2012, Stranic2012a, Stranic2013,
Rosado-Reyes2012a, Rosado-Reyes2012}. At Stanford University, researchers
measured the time-history of the concentration of the fuel, OH, H$_2$O, C$_2$H$_4$, CO, and CH$_4$
behind reflected shock waves for \textit{n}-, \textit{s}-,
and \iBuOH{} \cite{Cook2012, Stranic2012a, Stranic2013}.
\textcite{Rosado-Reyes2012a, Rosado-Reyes2012} measured the thermal
decomposition of \textit{n}- and \sBuOH{} in a
single-pulse ST and derived rate expressions for the decomposition
reactions.

\textcite{Heufer2011} reported high pressure ignition delay
results of stoichiometric \nBuOH{}/air mixtures under the
conditions behind the reflected shock of approximately \SIrange{10}{42}{\bar}
and \SIrange{770}{1250}{\kelvin}. The results of \textcite{Heufer2011}
showed an interesting non-Arrhenius behavior at
temperatures lower than about \SI{1000}{\kelvin} for the pressure range
studied. They found that the rate of increase of ignition delay with
decreasing temperature appeared to change around \SI{1000}{\kelvin}.
\textcite{Vranckx2011} further developed the low-temperature oxidation
mechanism of \nBuOH{} by performing experiments between
\SIrange{61}{92}{\bar} and \SIrange{795}{1200}{\kelvin} and updating
a kinetic model with a butyl-peroxy sub-mechanism. They showed improved
agreement with predictions of low-temperature butanol ignition delays,
but incorrectly predicted the existence of two-stage ignition phenomena.

\textcite{Zhu2014} measured the ignition delays of \nBuOH{}
in a ST using a newly developed technique known as constrained
reaction volume (CRV). In traditional ST experiments,
it is difficult to measure ignition delays longer than approximately
\SIrange{1}{10}{\milli\second} because fluid-dynamic effects and other
phenomena invalidate the assumptions typically used to calculate the
thermodynamic state. In the CRV strategy, the reactants are effectively
limited to a small region in the ST ensuring that the conditions
under which ignition occurs are constant enthalpy/nearly constant pressure
and are well characterized for longer time
scales than in traditional ST experiments. \textcite{Zhu2014}
were thus able to measure ignition delays of \nBuOH{} between
temperatures of \SIrange{716}{1121}{\kelvin}, pressures of
\SIlist{20;40}{\atmosphere}, and equivalence ratios of
$\phi = \numlist{0.5;1.0;2.0}$. Using the CRV strategy and constant
enthalpy/constant pressure modeling assumptions, \textcite{Zhu2014}
demonstrated that one recent kinetic model was able to accurately predict the
ignition delay of \nBuOH{} for most of the conditions
they studied.

Ignition delay experiments of the butanol isomers have also been
performed in RCMs. \textcite{Weber2011} studied the ignition delays of
\nBuOH{} for low- to intermediate-temperature conditions between
\SIrange{675}{925}{\kelvin}, pressures of
\SIlist{15;30}{\bar}, and equivalence ratios of $\phi = \numlist{0.5;1.0;2.0}$.
\textcite{Weber2011} found no evidence of two-stage ignition or
non-Arrhenius behavior in their results. \textcite{Weber2011} also found
that models available until the time of their work (2011) were unable
to predict the ignition delays of \nBuOH{}, over-predicting
the ignition delay by approximately one order of magnitude. Subsequently,
\textcite{Weber2013} extended their study to the other isomers of
butanol, covering temperatures between \SIrange{715}{910}{\kelvin},
pressures of \SIlist{15;30}{\bar}, and the stoichiometric equivalence
ratio. Results from the study by \textcite{Weber2013} are presented in
\cref{chap:buoh}. In summary, \textcite{Weber2013} found that the order of
reactivity---in terms of the inverse of ignition delay---of the butanol
isomers changed when the pressure was changed from \num{15} to \SI{30}{\bar}.
Moreover, \textcite{Weber2013} found unique pre-ignition heat release
behavior during the ignition of \tBuOH{} that was not present
during the ignition of the other isomers.

\textcite{Weber2013a} studied the autoignition of \iBuOH{}
at three mixture conditions, including $\phi = 0.5$ with air as the
oxidizer and $\phi = \numlist{0.5;2.0}$ where the O$_2$:N$_2$
ratio in the oxidizer was changed while the fuel mole fraction was held constant
to change the equivalence ratio. \textcite{Weber2013a} found that
a newly developed kinetic model for \iBuOH{} combustion
was able to predict the stoichiometric (from the work of \textcite{Weber2013})
and lean ignition delays in air, but was unable to capture the
dependence of the ignition delays on the initial oxygen concentration.
In addition, \textcite{Weber2011, Zhu2014} noted similar inability to predict
the dependence of ignition delay on initial oxygen concentration for
\nBuOH{} for several different kinetic mechanisms.

\textcite{Karwat2011a} studied the ignition delays of \nBuOH{}
for stoichiometric mixtures over temperatures from \SIrange{920}{1040}{\kelvin}
and pressures near \SI{3}{\atmosphere}. \textcite{Karwat2011a} found good
agreement of the ignition delays with the kinetic model developed in the
study of \textcite{Black2010}. In addition, \textcite{Karwat2011a} used
a high-speed sampling valve to remove gas samples from the reaction chamber
during the induction period of \nBuOH{} ignition. They quantified
mole fractions of CH$_4$, CO, C$_2$H$_4$, C$_3$H$_6$, C$_2$H$_4$O, C$_4$H$_8$O,
1-C$_4$H$_8$, and \nBuOH{} at several times during the ignition.
Comparison of the time histories of these species with predictions from
the model by \textcite{Black2010} showed that, although the model was able
to predict the ignition delay well, it was not able to reproduce the time history
of species concentrations very well, particularly C$_2$H$_4$. This result
demonstrates the importance of rigorously validating a kinetic model over
a wide range of conditions and for a wide range of validation targets.

In comparison to the butanol isomers, \iPeOH{} has received significantly
less focus in the literature. Studies of the combustion of \iPeOH{}
have been conducted in jet stirred reactors (JSRs) \cite{Dayma2011,Togbe2011,Sarathy2013}, low-pressure flow
reactors \cite{Welz2012}, counterflow flame experiments \cite{Sarathy2013},
STs \cite{Sarathy2013, Tsujimura2012, Tang2013}, and RCMs
\cite{Sarathy2013, Tsujimura2012}. Other studies have investigated
the efficacy of using \iPeOH{} in a homogeneous charge compression iginition (HCCI) engine \cite{Tsujimura2011,
Yacoub1998, Yang2010}. Finally, studies described in this work have
been conducted to determine the ignition properties of \iPeOH{} (see \cref{chap:peoh}).
Both the works by \textcite{Tsujimura2012, Sarathy2013} developed
detailed kinetic models for the combustion of \iPeOH{} whose validation
was based, in part, on ignition delay experiments. Using STs and
RCMs in concert, these studies were able to provide ignition delays
for temperatures, pressures, and equivalence ratios of
\SIrange{650}{1450}{\kelvin}, \SIrange{7}{60}{\bar}, and $\phi =
\numlist{0.5;1.0;2.0}$, respectively. These studies generally found good
agreement of their models with their validation data sets, although
\textcite{Sarathy2013} found that their model had difficulty predicting
rich ignition delays. In addition, substantial pre-ignition heat release
was observed for all of the equivalence ratios at \SI{40}{\bar} in the
RCM measurements, similar to \tBuOH{}.

\section{Recent Work on Ignition of Methylcyclohexane}

Several studies have suggested the use of methylcyclohexane (MCH) as a
component in surrogate formulations \cite{Bieleveld2009,Naik2005}, as
discussed previously. Furthermore, MCH is the simplest branched or
substituted cycloalkane, and can therefore provide a base from which
to build models of the combustion of other, larger naphthenes.

Substantial experimental and modeling work has been conducted for
napthenes in general, and MCH in particular. \textcite{Pitz2011}
conducted an extensive review of the work on naphthenes, so only studies
involving homogeneous ignition of MCH are discussed here. Ignition delays of MCH
have been measured in STs \cite{Rotavera2013, Vasu2009,
Vanderover2009, Hawthorn1966, Orme2006, Hong2011} and RCMs \cite{Tanaka2003,
Pitz2007, Mittal2009} by a number of researchers. These studies collectively
cover the temperature-pressure space in the range of \SIrange{700}{2100}{\kelvin}
and \SIrange{1}{70}{\atmosphere}. To complement this experimental work, a number
of kinetic models for MCH combustion have been constructed, notably by
\textcite{Orme2006, Pitz2007}.

The study of \textcite{Rotavera2013} measured ignition delays of MCH behind
reflected shock waves near \SIlist{1;10}{\atmosphere} for equivalence
ratios of $\phi = \numlist{0.5;1.0;2.0}$. They compared their measured ignition
delays with predictions from the model of \textcite{Pitz2007} and found
generally good agreement. \textcite{Hong2011} measured ignition delays
for conditions of temperature between \SIrange{1280}{1480}{\kelvin}, pressures
of \SIlist{1.5;3}{\atmosphere}, and equivalence ratios of $\phi = \numlist{1.0;0.5}$.
\textcite{Hong2011} compared their measurements with three mechanisms
from the literature, including those by \textcite{Pitz2007, Orme2006}
and found relatively good agreement for their conditions.

However, other studies have found that the existing models are not able
to predict ignition delays at conditions for which they were not validated%
---that is, the models are not truly predictive. For instance, previous work
conducted in an RCM by \textcite{Mittal2009} measured the ignition delays
of MCH/O$_2$/N$_2$/Ar mixtures at pressures of \SIlist{15.1;25.5}{\bar},
for three equivalence ratios of $\phi = \numlist{0.5;1.0;1.5}$, and over
the temperature range of \SIrange{680}{840}{\kelvin}. They compared
their measured ignition delays to simulated ignition delays computed
using the mechanism of \textcite{Pitz2007} and found that the model
substantially over-predicted both the first stage and overall ignition delay
\cite{Mittal2009}. Moreover, studies conducted in STs by
\textcite{Vasu2009, Vanderover2009} came to similar conclusions, which
collectively considered conditions between \SIrange{795}{1560}{\kelvin}
and \SIrange{1}{70}{\atmosphere}. Further studies described in \cref{chap:mch}
and published in the work of \textcite{Weber2014} have expanded the
validation range of MCH ignition data and substantially improved the
predictive ability of kinetic models of MCH combustion.

\section{Gas Sampling in Rapid Compression Machines}

Due to its relevance in predicting the performance of a fuel in existing
and advanced engines, ignition delay is a very common measure of the
global performance of a kinetic mechanism. Ignition delays for homogeneous
systems are typically measured in STs or RCMs, where the effects
of fluid motion and turbulence are generally minimized. However, as demonstrated
for the case of butanol isomers, \iPeOH{}, and methylcyclohexane, the
validation target of ignition delays is necessary but not sufficient to
develop truly predictive kinetic models.

Optical methods can offer non-intrusive in-situ measurements of species
and temperature during homogeneous ignition events---c.f.\ the work by
\textcite{Das2012a, Uddi2012} to directly measure the temperature and
water number density in the reaction chamber of an RCM using mid-IR
laser light absorption and the work by \textcite{Stranic2013} to perform
simultaneous concentration measurements of multiple species in their
ST. However, these methods can detect only a limited set of
species and require extensive calibration at engine-relevant pressure
conditions.

Another avenue to improve the rigor of validation targets is to remove
samples from the reacting gas and analyze them ex-situ. Work on this
avenue began in the early part of the 20th century, to help explain the
phenomenon of ``knock'' in engines. Several researchers developed
techniques to remove gas samples from the cylinder of spark-ignition
engines. According to \textcite{Withrow1930}, \textcite{Brooks1922} was
the first to develop a system to withdraw samples from the cylinder of
an oil-injection engine around 1922. Subsequently, \textcite{Callendar1926,
Egerton1926, Lovell1927, Ricardo1928, Withrow1930, Steele1933, Egerton1935,
Downs1951, Pahnke1954} further developed these systems for sampling from
the cylinder of spark-ignition engines.

In 1961, \textcite{Roblee1961} adapted a sampling device to an RCM for
the first time. The design of this sampling apparatus was such that the
entire reaction chamber could be quickly evacuated to an expansion chamber
through a punctured diaphragm. The diaphragm was ruptured either by
pressure difference between the reaction chamber and the expansion
chamber, or by a spring-actuated knife. Upon diaphragm rupture, the
gases in the reaction chamber rushed into the expansion chamber,
generating a shock wave that propagated further into the expansion
chamber. Simultaneously, a rarefaction wave was generated that
propagated backwards into the combustion chamber, expanding the gases
therein and quenching any ongoing reactions. After quenching, the
products were transferred to a gas chromatograph for analysis.

\textcite{Roblee1961} used this apparatus to study the decomposition of
benzene during the induction period. \textcite{Roblee1961} noted that minimal
consumption of benzene and oxygen occurred during the induction period.
Using a similar technique, but in a different RCM, \textcite{Martinengo1965}
measured the products of the decomposition of \textit{i}-octane and
\textit{n}-octane at temperatures ranging from \SIrange{600}{700}{\kelvin}
and pressures between \SIrange{15}{20}{\atmosphere}. \textcite{Martinengo1965}
noted that the main intermediate species they measured were alkenes and carbonyl
compounds under these conditions, and that CO was produced nearly simultaneously
with the final stage of ignition.

\textcite{Affleck1967} used a wide-aperture, electronically-triggered valve
to effect sampling from their RCM. The valve was triggered after the first
stage of ignition of 2-methylpentane, and the sample was quenched by
adiabatic cooling through expansion into a large sampling chamber.
\textcite{Affleck1967} also compared the composition of the samples from
high-pressure experiments in the RCM with samples drawn from low-pressure
ignition experiments conducted in Pyrex bulbs. The authors noted that
the products were largely similar between the two experiments, despite the
wide variation in pressure.

\textcite{Beeley1980} used the diaphragm-puncture method to analyze the
reaction intermediates during the pyrolysis and combustion of isopropyl
nitrate. The results from the pyrolysis indicated that the breakdown
of the fuel did not lead to chain branching and thus ignition, whereas
when oxygen was added, chain branching pathways were available and hot
ignition was observed.

The group at the Université des Sciences et Technologies de Lille in France
has conducted a number of studies using a sampling apparatus fitted to
their RCM \cite{Minetti1994, Minetti1995, Minetti1996, Ribaucour1998,
Minetti1999, Ribaucour2000a, Roubaud2000a, Lemaire2001, Ribaucour2002,
Vanhove2006a, Vanhove2006, Crochet2010}. These studies have measured the
concentration of intermediate species during the ignition of \textit{n}-butane,
\textit{n}-heptane, \textit{i}-octane, \textit{n}-pentane, 1-pentene,
\textit{o}-xylene, \textit{o}-ethyltoluene, \textit{n}-butylbenzene,
1-hexene, \textit{n}-propylcyclohexane, toluene, cyclohexane, cyclohexene,
and cyclohexa-1,3-diene. The sampling system used in these studies
is similar to that developed by \textcite{Roblee1961} in that it uses an
expansion chamber separated from the reaction chamber by a diaphragm
that is punctured by a knife at the appointed time.

Using a gas chromatograph coupled to a mass spectrometer to analyze their
samples, \textcite{Minetti1994} were able to identify approximately 25
species produced during the autoignition of \textit{n}-butane. These
species included several cyclic ethers produced during the low-temperature
oxidation process through peroxy species. \textcite{Minetti1994} noted
that a kinetic model for the combustion of \textit{n}-butane was able
to accurately reproduce the major species profiles, but was unable to
predict the mole fractions of several minor species.

\textcite{Minetti1995} studied the species produced during autoignition
of \textit{n}-heptane and found that, although a detailed kinetic model
was able to correctly predict the ignition delays, it was unable to
capture the concentration profiles of the major species, including
\textit{n}-heptane and its oxidation products. \textcite{Minetti1996}
compared the oxidation products of \textit{n}-heptane with
\textit{i}-octane under conditions of similar ignition delay. The
authors noted that the kinetic scheme for two-stage ignition for the
two fuels was similar in the sense that the important reaction classes
were the same; however, the species produced by the two fuels were
largely different. Moreover, \textcite{Minetti1996} emphasized the importance
of oxygenated species such as cyclic ethers and unsaturated hydrocarbons
in the low-temperature chain branching pathways.

\textcite{Ribaucour1998, Minetti1999} studied the autoignition of
\textit{n}-pentane and 1-pentene by analyzing the intermediate species
formed during the induction period. They further emphasized the importance
of cyclic ether and ketone formation in the low-temperature ignition process.
\textcite{Ribaucour1998, Minetti1999} also note that the presence of the
double bond in 1-pentene causes marked differences in the selectivity of
the intermediate species compared to \textit{n}-pentane, although the
overall oxidation scheme of the fuels can be described by similar reaction types.

\textcite{Ribaucour2000a, Roubaud2000a} studied the autoignition of alkylated
aromatics including \textit{n}-butylbenzene, \textit{o}-xylene, and
\textit{o}-ethyltoluene. The authors found that a detailed oxidation
mechanism was able to well reproduce the ignition delays and intermediate
species profiles of those fuels. Moreover, similar intermediate products
as for alkane speces were noted for the aromatic species with ortho-alkyl
groups. These species are able to undergo the critical hydrogen-transfer
reactions in the low-temperature chain branching pathways due to the position
of the alkyl group. Other aromatic species such as toluene, \textit{m}-xylene,
and \textit{p}-xylene react through different low-temperature chain branching
pathways and do not have the typical two-stage ignition and NTC region
that alkanes and ortho-alkyl aromatics have.

\textcite{Lemaire2001, Ribaucour2002} studied the concentration of
intermediate species during autoignition of cyclohexane, cyclohexene,
and cyclohexa-1,3-diene. \textcite{Lemaire2001} noted that cyclohexane
was prone to the same autoignition phenomena as acyclic alkanes, namely
two-stage ignition at low temperatures, followed by a region of negative
temperature dependence of the ignition delay as the temperature increases,
and finally, single stage ignition at high temperatures. They further
observed that cyclohexa-1,3-diene did not exhibit such behavior.
\textcite{Lemaire2001} also compared the species present during the
autoignition and found that cyclohexane showed many of the same species
as acyclic alkane ignition, whereas cyclohexa-1,3-diene did not have
such species. Cyclohexene showed behavior intermediate between
cyclohexane and cyclohexa-1,3-diene, both in terms of the ignition behavior
and the species concentrations.

\textcite{Ribaucour2002} studied the autoignition of cyclohexene and
constructed a detailed model for its oxidation. They found two primary
reaction pathways, one involving the double bond and the other involving
peroxy radicals, during the ignition period and also noted that effects
from the ring structure and the double bond combined to produced the
observed chemistry.

\textcite{Vanhove2006a, Vanhove2006} studied binary blends of 1-hexene
and toluene with the primary reference fuels, \textit{n}-heptane and
\textit{i}-octane, and one ternary blend of \textit{i}-octane/1-hexene/toluene.
The authors found that each fuel primarily propagated through its own
pathways, without much direct interaction between the fuels in the mixture.
However, the fuels competed for the chain-branching radicals and the
addition of unsaturated species tended to have a stronger effect on the
reactivity in the NTC regime, where the saturated hydrocarbon reactivity
is limited by the NTC reactions but the unsaturated species can still
react by adding radicals to the double bond.

\textcite{Crochet2010} investigated autoignition of \textit{n}-propylcyclohexane.
They found that the fuel forms several bicyclic ethers and conjugated
alkenes during the induction period. These species are formed through low
temperature chaing branching pathways by reaction with both the alkyl
chain and the cyclohexyl ring in \textit{n}-propylcyclohexane.

\textcite{Mittal2006a} at Case Western Reserve University also developed
a sampling system for their RCM. The design was similar to the design of
\textcite{Roblee1961}. \textcite{Mittal2007} demonstrated the
feasibility of their sampling apparatus by measuring the major species
during the induction period of methane ignition. To date, no further
results have been presented from this sampling apparatus.

Finally, \textcite{He2005a} developed a unique sampling apparatus for the
Rapid Compression Facility (RCF) at the University of Michigan. This
sampling apparatus quenches only a small portion of the reactants from
the reaction chamber instead of quenching the entire chamber, as in the
design of \textcite{Roblee1961} and similar designs. In the design of
\textcite{He2005a}, a small diameter tube protrudes into the reaction
chamber through which samples are drawn into a large expansion chamber.
The sampling time is controlled by a fast-acting solenoid valve located
outside the reaction chamber that is triggered based on the position of
the piston in the RCF and a delay timer.

The local sampling technique developed by \textcite{He2005a} has several
important advantages compared to the global sampling techniques used
in previous work. In particular, capturing the entire reaction chamber
also captures the boundary layer near the chamber walls, potentially
causing significant dilution of the test sample. By capturing only a
small sample from the center of the reaction chamber, the boundary layer
and attendant dilution can be avoided. In addition, the local sampling
technique does not substantially disturb the ongoing reactions,
meaning that the ignition process is allowed to proceed unhindered.

Nonetheless, the local sampling technique has some disadvantages.
Notably, the presence of dead volume in the sampling system can impact
the quantification of species in a manner similar to the capture of the
boundary layer in global sampling techniques. In particular, the dead
volume will have a much lower temperature than the core gases due to the
large surface-area-to-volume ratio, which prevents reactions from
occurring in the dead volume.

\textcite{He2007} used this sampling system to study the oxidation
products developed during the autoignition of \textit{i}-octane and noted
that comparison to a detailed kinetic model developed after the work of
\textcite{Minetti1996} showed agreement within a factor of two for most
species. For the species that showed larger disagreement, the authors
were able to use their results to suggest several alternate oxidation
pathways that were not included in the model.

Subsequently, \textcite{Walton2008} upgraded the sampling system at the
University of Michigan to reduce the dead volume and improve the
response time. \textcite{Walton2011} used the upgraded system to study
the intermediate species in the oxidation of methyl butanoate. The
authors found that a kinetic model was able to well predict the concentration
profiles of the several species, although larger disagreement was noted
for propene.

\textcite{Karwat2011a, Karwat2012, Karwat2013} used the upgraded
sampling system to study the autoignition chemistry of \nBuOH{}
\cite{Karwat2011a}, \textit{n}-heptane/\textit{n}-butanol blends
\cite{Karwat2012}, and \textit{n}-heptane \cite{Karwat2013} (the results
for \nBuOH{} have been described previously). \textcite{Karwat2012}
demonstrated that the reactivity of \textit{n}-heptane was reduced when
blended with \nBuOH{}, and moreover that the fundamental reaction
pathways of \textit{n}-heptane were changed by the addition of \nBuOH{}.
A kinetic model over-predicted the consumption of \textit{n}-heptane
during the first stage of ignition, and was thus unable to reproduce many
of the species profiles for the duration of the induction period.
\textcite{Karwat2013} further used species sampling measurements of pure
\textit{n}-heptane, combined with newly calculated reaction rate constants
for alkylperoxy reactions from the literature, to improve a kinetic
model of \textit{n}-heptane combustion, although they did not compare the
updated model to their blending results.

\section{Summary}

The works presented in the previous sections represent a large volume
of validation data for kinetic models, and have greatly expanded our
understanding of the combustion chemistry of alternative fuels.
Nonetheless, gaps in the state of the art knowledge have been revealed
through several experimental studies, gaps that prevent the development
of truly predictive kinetic models---for example, the inability of
models to predict the oxygen concentration dependence of ignition delays
is still unexplained.

Thus, the major objectives of this work can be stated succinctly as
follows:

\begin{enumerate}
\item \label{item:1} Generate ignition delay datasets for alternative
fuels at high-pressure, low-temperature conditions that have not been
studied extensively in previous work

\item \label{item:2} Develop and characterize a new experimental
apparatus to enable ex-situ species measurements from the reaction
chamber of the RCM during the ignition delay

\item Use the data acquired from \cref{item:1} and \cref{item:2}
to extend the validation of new and existing chemical kinetic models
for the combustion of alternative fuels

\item Analyze new and existing chemical kinetic models to help
understand the cause of discrepancies and clarify the important
reaction pathways in high-pressure alternative fuel ignition
\end{enumerate}

\section{Organization of this Work}

The bulk of what follows has been published in the archival literature,
and the preceding introduction has primarily been taken from those
papers (and includes a summary of some results from the following
chapters). In addition, much of \cref{chap:facilities} has been
taken from the published work. \cref{chap:facilities} presents an
introduction to the RCM and other facilities used in the experiments
described in subsequent chapters, including the fast sampling system
that was newly developed for this work. Detailed uncertainty analyses
are also considered for the appropriate apparatuses.

The subsequent chapters are organized by the fuel studied:
\cref{chap:buoh} considers the butanol isomers and was published in
\textit{Energy \& Fuels} \cite{Weber2013} and the 8th U.S. National
Combustion Meeting \cite{Weber2013a}; \cref{chap:peoh} considers
\iPeOH{} and was published in \textit{Combustion and Flame}
\cite{Sarathy2013}; \cref{chap:mch} considers methylcyclohexane and was
published in \textit{Combustion and Flame} \cite{Weber2014}. Finally,
\cref{chap:conclusions} presents conclusions based on this work and
recommendations for future directions.

The works published in or submitted to the archival literature during the
course of this program are as follows:
\begin{enumerate}
\item[] Weber, B.W., Kumar, K., Zhang, Y., and Sung, C.-J.
        ``Autoignition of \nBuOH{} at elevated pressure and
        low-to-intermediate temperature.'' \textit{Combust. Flame},
        vol. 158, no. 5 (Mar. 2011), pp. 809--819.
        DOI: \doi{10.1016/j.combustflame.2011.02.005}.
\item[] Tsujimura, T., Pitz, W. J., Gillespie, F., Curran, H. J., Weber,
        B. W., Zhang, Y., and Sung, C.-J. ``Development of Isopentanol
        Reaction Mechanism Reproducing Autoignition Character at High
        and Low Temperatures.'' \textit{Energy Fuel}, vol. 26, no. 8
        (Aug. 2012), pp. 4871--4886. DOI: \doi{10.1021/ef300879k}.
\item[] Weber, B. W. and Sung, C.-J. ``Comparative Autoignition Trends
        in Butanol Isomers at Elevated Pressure.'' \textit{Energy Fuel},
        vol. 27, no. 3 (Mar. 2013), pp. 1688--1698.
        DOI: \doi{10.1021/ef302195c}.
\item[] Sarathy, S. M., Park, S., Weber, B. W., Wang, W., Veloo, P. S.,
        Davis, A. C., Togbé, C., Westbrook, C. K., Park, O., Dayma, G.,
        Luo, Z., Oehlschlaeger, M. A., Egolfopoulos, F. N., Lu, T.,
        Pitz, W. J., Sung, C.-J., and Dagaut, P. ``A comprehensive
        experimental and modeling study of iso-pentanol combustion.''
        \textit{Combust. Flame}, vol. 160, no. 12 (Dec. 2013),
        pp. 2712--2728. DOI: \doi{10.1016/j.combustflame.2013.06.022}.
\item[] Weber, B. W., Pitz, W. J., Mehl, M., Silke, E. J., Davis, A. C.,
        and Sung, C.-J. ``Experiments and modeling of the autoignition
        of methylcyclohexane at high pressure.'' \textit{Combust. Flame},
        (Feb. 2014), In Press.
        DOI: \doi{10.1016/j.combustflame.2014.01.018}.
\item[] Burke, S., Burke, U., Mathieu, O., Osorio, I., Keesee, C.,
        Morones, A., Petersen, E. L., Wang, W., DeVerter, T.,
        Oehlschlaeger, M. A., Rhodes, B., Hanson, R. K., Davidson, D. F.,
        Weber, B. W., Sung, C.-J., Santner, J., Ju, Y., Haas, F. M.,
        Dryer, F. L., Volkov, E., Nilsson, E., Konnov, A., Alrefae, M.,
        Khaled, F., Farooq, A., Dirrenberger, P., Glaude, P.-A., and
        Battin-Leclerc, F. ``An Experimental and Modeling Study of
        Propene Oxidation. Part 2: Ignition Delay Time and Flame Speed
        Measurements.'' \textit{Combust. Flame}, (Submitted).
\end{enumerate}
\end{document}
