% arara: xelatex: { synctex: on, shell: off }
% arara: biber
% arara: xelatex: { synctex: on, shell: off }
% arara: sumatrapdf
\documentclass[../main.tex]{subfiles}

% %Set the package to import preambles
\usepackage{subfiles}

%Load graphicx here to specify options
\usepackage[final]{graphicx}

%Set the document font
\usepackage[no-math]{fontspec}
\setmainfont[Ligatures=TeX]{Times New Roman}
\setmonofont{Inconsolata}

%Set the text to double spacing
%According to hyperref README,
%setspace should be loaded first
\usepackage[doublespacing]{setspace}

%Set a command to easily skip a line
\newcommand{\blankline}{\vspace*{\baselineskip}}

%Set up biblatex
\usepackage[
    backend=biber,
    % url=false,
    doi=true,
    sorting=none,
    sortcites=true,
    maxbibnames=6,
    minbibnames=6,
    maxcitenames=2,
    mincitenames=1,
    citestyle=numeric-comp,
    firstinits=true,
    isbn=false
]{biblatex}
\addbibresource{C:/Users/\user/Documents/Github/dissertation/library.bib}

%Remove the "In:" from before the journal title for articles
\renewbibmacro{in:}{%
  \ifentrytype{article}{}{\printtext{\bibstring{in}\intitlepunct}}}

%Change the name of the bibliography section to "References"
\DefineBibliographyStrings{english}{bibliography = {References}}

%Set the sort order of the names in each bibliography entry
\DeclareNameAlias{default}{last-first}

%Don't print the article title. To print the title, add #1 to the last {}
\DeclareFieldFormat[article,incollection,unpublished]{title}{}

%Add "vol." and "no." before volume and issue.
\DeclareFieldFormat[article]{volume}{\bibstring{volume}\addspace #1}
\DeclareFieldFormat[article]{number}{\bibstring{number}\addspace #1}

%Ensure that a comma follows abbreviated journal titles.
\DeclareFieldFormat{journaltitle}{\mkbibemph{#1}\isdot}

%Put a comma between the volume and issue instead of period.
\renewbibmacro*{volume+number+eid}{%
  \printfield{volume}%
  \setunit{\addcomma\space}%<---- was \setunit*{\adddot}%
  \printfield{number}%
  \setunit{\addcomma\space}%
  \printfield{eid}}

%Add a comma after the journal title.
\renewbibmacro*{journal+issuetitle}{%
  \usebibmacro{journal}%
  \setunit*{\addcomma\addspace}%<---- was \setunit*{\addspace}%
  \iffieldundef{series}
    {}
    {\newunit
     \printfield{series}%
     \setunit{\addspace}}%
  \usebibmacro{volume+number+eid}%
  \setunit{\addspace}%
  \usebibmacro{issue+date}%
  \setunit{\addcolon\space}%
  \usebibmacro{issue}%
  \newunit}

%Only print URL if doi is not present.
\DeclareFieldFormat{url}{%
  \iffieldundef{doi}{%
    \mkbibacro{URL}\addcolon\space\url{#1}%
  }{%
  }%
}
\DeclareFieldFormat{urldate}{%
  \iffieldundef{doi}{%
    \mkbibparens{\bibstring{urlseen}\space#1}%
  }{%
  }%
}

%Remove publisher from being printed.
\renewbibmacro*{publisher+location+date}{%
  \printlist{location}%
  \setunit*{\addcomma\space}%
  \usebibmacro{date}%
  \newunit}

%Fix in-text full citations
\DeclareCiteCommand{\fullcite}
  {\usebibmacro{prenote}}
  {\usedriver
     {\defcounter{minnames}{99}%
      \defcounter{maxnames}{99}}
     {\thefield{entrytype}}}
  {\multicitedelim}
  {\usebibmacro{postnote}}

%Use fancy tables.
\usepackage{booktabs}

%Set up todo notes in the PDF file
\usepackage{todonotes}

%Use and set up the caption package for nicer captions.
\usepackage{caption}
\DeclareCaptionLabelFormat{bf}{\textbf{#1 #2}}
\captionsetup{
    font=small ,
    labelsep=colon ,
    labelformat=bf ,
    figurewithin=chapter ,
    tablewithin=chapter ,
}

\usepackage{titlesec}
\usepackage{titletoc}

\titleformat{\chapter}[display]{\normalfont\Huge\bfseries}{Chapter \thechapter}{0.7em}{}
\titleformat{\section}{\normalfont\LARGE\bfseries}{\thesection}{0.5em}{}
\titleformat{\subsection}{\normalfont\Large\bfseries}{\thesubsection}{1em}{}
\titleformat{\subsubsection}{\normalfont\large\bfseries}{\thesubsubsection}{1em}{}

\titlecontents{chapter}[0pc]{}{\bfseries Chapter \thecontentslabel\quad}{}{\titlerule*[0.5pc]{.}\contentspage}
\titlecontents{section}[1em]{}{\thecontentslabel\quad}{}{\titlerule*[0.5pc]{.}\contentspage}
\titlecontents{subsection}[2em]{}{\thecontentslabel\quad}{}{\titlerule*[0.5pc]{.}\contentspage}
\titlecontents{subsubsection}[3em]{}{\thecontentslabel\quad}{}{\titlerule*[0.5pc]{.}\contentspage}

\setcounter{secnumdepth}{3}
\setcounter{tocdepth}{3}

%Use the subfigure package
\usepackage{subfig}

%Various math improvements.
%Must be loaded before hyperref
\usepackage{mathtools}

%Set the math font. Has to come after mathtools because
%some font stuff gets overwritten.
\usepackage{unicode-math}
\unimathsetup{math-style=TeX}
\setmathfont[range=\mathup/{num}]{Times New Roman}
\setmathfont[range=\mathit/{greek,Greek,latin,Latin}]{Cambria Math}
\setmathfont[range=\mathup/{greek,Greek,latin,Latin}]{Cambria Math}
\setmathfont[range={"2212,"002B,"003D,"0028,"0029,"005B,"005D,"221A,
"2211,"2248,"222B,"007C,"2026,"2202,"00D7,"0302,"2261,"0025,"22C5,
"00B1,"2194,"21D4,"2260}]
{Cambria Math}

%Better looking fonts
\usepackage[final]{microtype}

%Allow table cells to span multiple rows.
\usepackage{multirow}

%Allow landscape rotated figures and captions.
\usepackage{afterpage}
\usepackage{rotating}
\usepackage{pdflscape}

%Set the root path where figures are stored.
\graphicspath{ {C:/Users/\user/Documents/Github/dissertation/figures/} }

%Set a convenience command for table cells that allow line breaks.
\newcommand{\linebreakcell}[2][c]{%
  \begin{tabular}[#1]{@{}c@{}}#2\end{tabular}}

%Use and set up the siunitx package for nice units printing.
\usepackage{siunitx}
\sisetup{%
    group-separator = {,},
    range-phrase = {\text{ to }},
    list-separator = {\text{, }},
    list-final-separator = {\text{, and }},
    list-pair-separator = {\text{ and }},
}%
\DeclareSIUnit\calorie{cal}
\DeclareSIUnit\atmosphere{atm}
\DeclareSIUnit\torr{torr}

%Declare convenience macros for printing the
%names of the alcohols.
\newcommand{\iPeOH}{\textit{i}-pentanol}
\newcommand{\nBuOH}{\textit{n}-butanol}
\newcommand{\sBuOH}{\textit{s}-butanol}
\newcommand{\tBuOH}{\textit{t}-butanol}
\newcommand{\iBuOH}{\textit{i}-butanol}

%The floatrow package allows multiple floats in a row
%and is set so that table captions are on top of the
%table.
\usepackage{floatrow}
\floatsetup[table]{style=plaintop}

%Use the titling package to allow easy access to custom title pages
\usepackage{titling}
\title{High Pressure Ignition Chemistry of Alternative Fuels}
\author{Bryan William Weber}

%Add bibliography and indices to the TOC
\usepackage{tocbibind}

%Improve handling of appendices
\usepackage{appendix}

%Use package that allows inline patching of commands. This is used in
%the appendices section.
\usepackage{xpatch}

%Use the bookmark package (which loads hyperref) so that only one
%compilation is necessary to get references.
\usepackage{bookmark}

%Set the color of the links and PDF metadata
\hypersetup{%
    pdfinfo={
        Title={High Pressure Ignition Chemistry of Alternative Fuels},
        Author={Bryan W. Weber},
    },
    colorlinks=true,
    citecolor=blue,
    linkcolor=black,
    plainpages=false,
    final,
}

%Allow lualatex to properly add links processed from pax files.
\usepackage{pdftexcmds}
\makeatletter
\let\pdfescapename=\pdf@escapename
\let\pdfstrcmp=\pdf@strcmp
\makeatother
\usepackage{pax}

%Allow to use \doi to link to DOI links.
\usepackage{doi}

%Allow inserting PDF documents directly to the output. According to
%http://tex.stackexchange.com/a/13660/32374, should come after hyperref
\usepackage{pdfpages}

%Do a better job with the automatic references. According to
%http://tex.stackexchange.com/a/1868/32374, should come after hyperref
\usepackage[capitalise, sort&compress]{cleveref}

%Set the auto-format names for the cleveref operations
\crefname{chapter}{Chapter}{Chapters}
\Crefname{chapter}{Chapter}{Chapters}
\crefname{section}{Sec.}{Secs.}
\Crefname{section}{Section}{Sections}
\crefname{subsection}{Sec.}{Secs.}
\Crefname{subsection}{Section}{Sections}
\crefname{subsubsection}{Sec.}{Secs.}
\Crefname{subsubsection}{Section}{Sections}
\crefname{figure}{Fig.}{Figs.}
\Crefname{figure}{Figure}{Figures}
\crefname{table}{Table}{Tables}
\Crefname{table}{Table}{Tables}
\crefname{equation}{Eq.}{Eqs.}
\Crefname{equation}{Equation}{Equations}
\crefname{appchap}{Appendix}{Appendices}
\Crefname{appchap}{Appendix}{Appendices}

\newcommand{\creflastconjunction}{, and~}
\newcommand{\crefrangeconjunction}{--}

%Set the size of the margins and the paper
%According to http://tex.stackexchange.com/a/26592/32374
%this should go after hyperref
\usepackage[margin=1in, letterpaper]{geometry}

%Set up the page numbers
%This has to go after geometry so the page number is centered
\usepackage{fancyhdr}
\pagestyle{fancy}
\fancyhf{}
\fancyfoot[C]{\thepage}
\renewcommand{\headrulewidth}{0pt}


\begin{document}
\label{sec:intro-introduction}

\section{Background}
\todo{Add lit review for sampling from RCM in general}

The world relies heavily on combustion to provide energy in useful forms for
human consumption; combustion currently represents over 80\% of the world
energy production \cite{Sims2007} and is predicted to decrease in importance
only slightly by 2040 \cite{EIA2013}. In particular, the transportation sector
accounts for nearly 40\% of the energy use in the United States and of that,
more than 90\% is supplied by combustion of fossil fuels \cite{MER2013}.
Unfortunately, emissions from the combustion of traditional fossil fuels have
been implicated in a host of deleterious effects on human health and the
environment \cite{Avakian2002} and fluctuations in the price of traditional
fuels can have a negative impact on the economy \cite{Owen2010}.

Despite its shortcomings, combustion is currently the only energy conversion
mechanism that offers the immediate capability to generate the sheer amount
of energy required to run the modern world. Since we cannot eliminate
combustion as an important energy conversion method, we must instead ameliorate
the shortcomings of a primarily combustion-based energy economy. A two-pronged
approach has developed to achieve the necessary improvements. These prongs
include: 1) development of new fuel sources and 2) development of new
combustion technologies. First, using new sources of fuel for combustion-based
energy conversion can reduce the economic impact of swings in the price of
current fuels, in addition to potentially reducing emissions. Second, using new
combustion technologies can reduce harmful emissions while simultaneously
increasing the efficiency of combustion processes, thereby reducing fuel
consumption.

Many new sources of fuels have been investigated recently. The most promising
of these in the long term are renewable biological sources, which are used to
produce fuels known as biofuels. The advantage of biofuels over traditional
fuels lies in their feedstocks. Whereas traditional fuel feedstocks generally
require millions of years to be produced, biofuel feedstocks are replenished
on an annual basis. Furthermore, biofuels offer the potential to offset carbon
emissions created from their combustion by reusing the emitted carbon to grow
the plants from which the fuels are produced.

However, the combustion
properties of biofuels may be substantially different from the traditional
fuels they are intended to replace. This makes it difficult to quickly switch
the energy economy to biofuels and necessitates medium-term investigation of
alternative sources for traditional fuels. These sources include shale oil and
liquefied coal, which have different chemical compositions than traditional
fuel sources and therefore fuels made from these alternative sources have
different combustion properties. Collectively, all of these fuels created from
non-traditional sources are known as alternative fuels.

In addition to new fuel sources, new engine technologies are rapidly being
developed. These include engines capable of operating in favorable combustion
regimes, such as so-called Low Temperature Combustion (LTC) engines and
Homogeneous Charge Compression Ignition (HCCI) engines. These devices avoid
regions in the temperature-equivalence ratio space where combustion generates a
large amount of emissions and operate in regions where efficiency is
maximized and emissions are reduced. Other devices, such as the well-known
catalytic converter, operate on the exhaust after it leaves the cylinder to
improve emissions characteristics.

Neither of these approaches is able to mitigate all of the negative impacts of
combustion by itself. By switching to biofuels but retaining the same engines,
the efficiency and emissions targets may not be met; by only developing new
engines, our sources of fuel will continue to cause economic distress, turmoil,
and negative effects on the environment. It will take a concerted effort to
bring these two pathways of innovation together.

Unfortunately, there are many roadblocks on the way to combining new fuels in
new engines. For instance, one can imagine the design and testing process of
new engines and fuels becoming circular: the ``best'' alternative fuel should be
tested in the ``best'' engine, but the ``best'' engine depends on which is selected
as the ``best'' alternative fuel. One way to cut this circle short is by
employing computer-aided design and modeling of new engines with new fuels to
design engines to be fuel-flexible. Accurate and predictive models of
combustion processes can be used to computationally test the efficacy of new
technologies and fuels before they undergo expensive real-world testing. The
key to this process is the development of accurate and predictive combustion
models.

Substantial work has been put forth recently to develop and validate predictive
combustion models for several alternative fuels. These studies include
calculation and measurement of reaction rate coefficients, measurement of
global and local combustion properties, and development of model construction
methodologies. Nevertheless, much of the work is still ongoing, and there is
substantial room for extending the state-of-the-art knowledge, especially at
high-pressure conditions relevant to combustion in engines.

A combustion model for the combustion of a given fuel in a given device
must necessarily accurately model the complete interaction of the operating
elements of the device. This includes sub-models for the chemical reactions that the
fuel and oxidizer undergo as well as the interaction of the fuel/oxidizer
mixture with the operation of the device. The first of these is known as
a chemical kinetic model or reaction mechanism; the second typically includes
such effects as turbulence interaction, heat transfer, liquid spray
dynamics, fluid mechanics, etc., each of which are typically modeled independently.

Chemical kinetic models for the combustion of large molecules are typically
built in a hierarchical fashion, as described by \textcite{Westbrook1984}. That
is, the model for the combustion of heptane contains the model for the
combustion of hexane added to the model of combustion for pentane, and so on
down to the models for hydrogen and carbon monoxide combustion. Therefore, it
is important to thoroughly validate the models for smaller species while
building models of higher hydrocarbons and other molecular types. Work has been
ongoing to explore the chemistry of small molecules for decades. Notable recent
kinetic mechanisms to emerge from this work include the GRI-Mech series of
mechanisms (most recently, version 3.0 \cite{Smith}), USC-Mech v2
\cite{Wang2007}, and the AramcoMech series of mechanisms, most recently version
1.3 \cite{Metcalfe2013}.

In addition, validation of kinetic models for the combustion of larger fuels
has proceeded in parallel with the small molecule chemistry. Given their projected
importance to combustion, one focus of the larger molecule work has naturally
been on biofuels. These biofuels typically include chemical species such as
alcohols and esters – neat alcohols can be used as fuels, while esters are
typically found as components of biodiesel fuels. A review by \textcite{Kohse-Hoinghaus2010}
covers much of the experimental data available until 2010. Since then an
enormous amount of data has been produced for both alcohols and esters.
Since the focus of this study is on alcohols, I will highlight alcohols
in the following sections.

Model construction and validation has also been focused on alternative ``traditional''
fuels, that is, fuels that are chemically similar to traditional fuels but
produced from alternative sources such as shale oil or coal liquefaction.
Traditional fuels and alternative ``traditional'' fuels typically contain
hundreds or thousands of chemical components. This makes building and using
models containing every species present in the fuel intractable on current
computer hardware. Therefore, a more useful approach to building models for
these fuels is to define a surrogate fuel. Surrogate fuels are made of a
limited number of chemical components to ensure that model building and use
are tractable, but the components are chosen so that the surrogate fuel
faithfully reproduces the physical and chemical properties of the real fuel.

Much progress has been made recently to construct surrogates for typical
transportation fuels. For instance, work on diesel surrogate formulation has
recently been reviewed by \textcite{Pitz2011}, work on gas turbine fuel
surrogates has been briefly summarized by \textcite{Dooley2012}, and work on
gasoline surrogates has been summarized in the work of \textcite{Anand2011,Pitz2007}.
One typical component class in the surrogate formulations is a cycloalkane or
alkyl-cycloalkane (collectively known as naphthenes), due to this class' presence
in nearly all transportation fuels \cite{Pitz2007, Briker2001, Farrell2007,
Edwards2007}. One particular cycloalkane, methylcyclohexane (MCH),
has been suggested in several surrogate formulations, including those by
\textcite{Bieleveld2009,Naik2005}. Recent work on MCH combustion will also
be highlighted in the following sections.

\section{Recent Work on the Combustion of Alcohols}

Among the alcohols being considered as biofuels, two criteria are typically
used to judge the suitability of a species: 1) its ease of production and
2) its potential as a ``drop-in'' replacement for current fuels. Because
of these criteria (among others), much research recently has focused on
the isomers of butanol, the C$_4$ alcohols, and \iPeOH{}, a C$_5$
alcohol. This is because these fuels are easy to produce by a number of biological
pathways \cite{Peralta-Yahya2012} and offer similar properties as
gasoline for use in typical automotive transportation applications \cite{Afeefy2014,
Davis2013}.

One of the most common biofuels currently in use is ethanol (C$_2$H$_6$O).
Although ethanol is ubiquitous at gasoline pumps, it suffers from several
disadvantages that suggest it needs to be replaced \cite{Niven2005}. In particular, ethanol
has a much lower energy density than gasoline, reducing volumetric fuel
economy, and ethanol is typically produced from crops that would otherwise
be used as food sources \cite{Somma2009}.

\textit{n}-Butanol has recently been identified as one of a suite of so
called ``second generation'' biofuels intended to supplement or replace
the ``first generation'' biofuels currently in use, such as ethanol \cite{Harvey2011,Nigam2011}.
The second generation biofuels will help alleviate some of the problems
identified with the first generation biofuels, including concerns about
feedstocks. In addition to the normal (\textit{n}) isomer, there are three
other isomers of butanol--- \textit{s}-, \textit{i}-, and \tBuOH{}.
Biological production pathways have been identified for \textit{n}-, \textit{s}-,
and \iBuOH{} \cite{Nigam2011,Smith2010}, but \tBuOH{} is a petroleum derived product.
Nevertheless, \tBuOH{} is currently used as an octane enhancer in gasoline.

In the last five years, research into the combustion characteristics of
the isomers of butanol has exploded, so exemplary references are
provided here except for the articles of particular interest to this work.
In addition to applied engine research \cite{Dernotte2009, Szwaja2010, Kim2011},
fundamental combustion measurements have been made using many different
systems. These include laminar flame speeds \cite{Veloo2011a}, jet-stirred reactor
chemistry \cite{Dagaut2009}, low-pressure flame structure \cite{Hansen2011a,Hansen2013},
atmospheric pressure flame structure \cite{Grana2010}, pyrolysis \cite{VanGeem2010, Cai2012, Cai2013},
flow reactors \cite{Lefkowitz2012,Heyne2013}, and ignition delays, which will be discussed
in more detail shortly. Other researchers have measured or calculated the reaction
rate constants of reactions of butanol with various radicals, including OH \cite{Stranic2013a,
Pang2012, Pang2012a, Seal2013, Pang2012b, El-Nahas2012, Zhou2011, Moc2010, Vasu2010}, HO$_2$
\cite{Zhou2012, Alecu2012, Black2010a}, and CH$_3$ \cite{Katsikadakos2013, Katsikadakos2012}.

Several studies of ignition delay of the butanol isomers have been conducted
in both shock tubes and RCMs, including work in shock tubes by \textcite{Moss2008,
Black2010, Noorani2010, Zhang2012, Stranic2012, Yasunaga2012, Heufer2011,
Vranckx2011, Zhu2014} and work in RCMs by \textcite{Weber2011,
Weber2013, Weber2013a, Karwat2011a}. These studies have covered a wide range of
temperature-pressure regimes, from \SIrange{1}{90}{\bar} and \SIrange{675}{1800}{\kelvin}.

Among the shock tube ignition studies, \textcite{Moss2008} have done
measurements for all four isomers of butanol at \SIlist{1;4}{\bar} and
\SIrange{1200}{1800}{\kelvin}, over equivalence ratios of $\phi = \numlist{0.5;1.0;2.0}$
and fuel mole percents of \SIlist{0.25;0.5;1.0}{\percent}.
\textcite{Black2010} investigated autoignition for \nBuOH{} from
\SIrange{1100}{1800}{\kelvin} and \SIlist{1;2.6;8}{\atmosphere} over $\phi = \numlist{0.5;1.0;2.0}$
and fuel mole percentages of \SIlist{0.6;0.75;3.5}{\percent}.
\textcite{Noorani2010} investigated the ignition delays of the primary alcohols
from C$_1$--C$_4$ at pressures of \SIlist{2;10;12}{\atmosphere} under dilute
conditions for equivalence ratio $\phi = \numlist{0.5;1.0;2.0}$, and temperatures
from \SIrange{1070}{1760}{\kelvin}. \textcite{Zhang2012} measured ignition
delays of \nBuOH{} at pressures of \SIlist{2;10}{\atmosphere},
temperatures in the range of \SIrange{1200}{1650}{\kelvin}, and for equivalence
ratios of \numlist{0,5;1.0;2.0}. \textcite{Stranic2012} measured ignition
delays of all four isomers of butanol over the pressure range \SIrange{1.5}{43}{\atmosphere},
temperature range \SIrange{1050}{1600}{\kelvin}, and equivalence ratios of
\numlist{0.5;1.0}. These studies showed generally good agreement of
the ignition delays for \nBuOH{}, but \textcite{Stranic2012}
found that their ignition delays for the other isomers of butanol were
shorter than the ignition delays measured by \textcite{Moss2008}.
\textcite{Stranic2012} were unable to determine the reason for the
discrepancy.

\textcite{Yasunaga2012} measured ignition delays of \textit{s}-,
\textit{t}-, and \iBuOH{} a pressure of \SI{3.5}{\atmosphere}
and temperatures between \SIrange{1250}{1800}{\kelvin}. In addition,
\textcite{Yasunaga2012} measured reactant, intermediate, and product
species during pyrolysis of all four butanol isomers by sample extraction
from their shock tube and analysis by gas chromatography. Other researchers
have measured species profiles during the pyrolysis of butanol isomers
in a shock tube by optical techniques, including \textcite{Cook2012, Stranic2012a, Stranic2013,
Rosado-Reyes2012a, Rosado-Reyes2012}. At Stanford University, researchers
measured the time-histories of the fuel, OH, H$_2$O, C$_2$H$_4$, CO, and CH$_4$
were measured behind reflected shock waves for \textit{n}-, \textit{s}-,
and \iBuOH{} \cite{Cook2012, Stranic2012a, Stranic2013}.
\textcite{Rosado-Reyes2012a, Rosado-Reyes2012} measured the thermal
decomposition of \textit{n}- and \sBuOH{} in a
single-pulse shock tube and derived rate expressions for the decomposition
reactions.

\textcite{Heufer2011} reported high pressure ignition delay
results of stoichiometric \nBuOH{}/air mixtures under the
conditions behind the reflected shock of approximately \SIrange{10}{42}{\bar}
and \SIrange{770}{1250}{\kelvin}. The results of \textcite{Heufer2011}
showed an interesting non-Arrhenius behavior at
temperatures lower than about \SI{1000}{\kelvin} for the pressure range
studied. They found that the rate of increase of ignition delay with
decreasing temperature appears to change around \SI{1000}{\kelvin}.
\textcite{Vranckx2011} further developed the low-temperature oxidation
mechanism of \nBuOH{} by performing experiments between
\SIrange{61}{92}{\bar} and \SIrange{795}{1200}{\kelvin} and updating
a kinetic model with a butyl-peroxy sub-mechanism. They showed improved
agreement with predictions of low-temperature butanol ignition delays,
but incorrectly predicted the existence of two-stage ignition phenomena.

\textcite{Zhu2014} measured the ignition delays of \nBuOH{}
in a shock tube using a newly developed technique known as constrained
reaction volume (CRV). In traditional shock tube experiments,
it is difficult to measure ignition delays longer than approximately
\SIrange{1}{10}{\milli\second} because fluid-dynamic effects and other
phenomena invalidate the assumptions typically used to calculate the
thermodynamic state. In the CRV strategy, the reactants are effectively
limited to a small region in the shock tube ensuring that the conditions
under which ignition occurs are constant enthalpy/nearly constant pressure
and are well characterized for longer time
scales than in traditional shock tube experiments. \textcite{Zhu2014}
were thus able to measure ignition delays of \nBuOH{} between
temperatures of \SIrange{716}{1121}{\kelvin}, pressures of
\SIlist{20;40}{\atmosphere}, and equivalence ratios of
$\phi = \numlist{0.5;1.0;2.0}$. Using the CRV strategy and constant
enthalpy/constant pressure modeling assumptions, \textcite{Zhu2014}
demonstrated that one recent kinetic model was able to predict the
ignition delay of \nBuOH{} well for most of the conditions
they studied.

Ignition delay experiments of the butanol isomers have also been
performed in rapid compression machines (RCMs). \textcite{Weber2011}
studied the ignition delays of \nBuOH{} for low- to intermediate-%
temperature conditions between \SIrange{675}{925}{\kelvin}, pressures of
\SIlist{15;30}{\bar}, and equivalence ratios of $\phi = \numlist{0.5;1.0;2.0}$.
\textcite{Weber2011} found no evidence of two-stage ignition or
non-Arrhenius behavior in their results. \textcite{Weber2011} also found
that models available until the time of their work (2011) were unable
to predict the ignition delays of \nBuOH{}, over-predicting
the ignition delay by approximately one order of magnitude. Subsequently,
\textcite{Weber2013} extended their study to the other isomers of
butanol, covering temperatures between \SIrange{715}{910}{\kelvin},
pressures of \SIlist{15;30}{\bar}, and the stoichiometric equivalence
ratio. Results from the study by \textcite{Weber2013} are presented in
\cref{chap:buoh}. In summary, \textcite{Weber2013} found that the order of
reactivity---in terms of the inverse of ignition delay---of the butanol
isomers changes when the pressure is changed from \num{15} to \SI{30}{\bar}.
Moreover, \textcite{Weber2013} found unique pre-ignition heat release
behavior during the ignition of \tBuOH{} that was not present
during the ignition of the other isomers.

\textcite{Weber2013a} studied the autoignition of \iBuOH{}
at three mixture conditions, including $\phi = 0.5$ with air as the
oxidizer and $\phi = \numlist{0.5;2.0}$ where the O$_2$:N$_2$
ratio in the oxidizer was changed while the fuel mole fraction was held constant
to change the equivalence ratio. \textcite{Weber2013a} found that
a newly developed kinetic model for \iBuOH{} combustion
was able to predict the stoichiometric (from the work of \textcite{Weber2013})
and lean ignition delays in air, but was unable to capture the
dependence of the ignition delays on the initial oxygen concentration.
In addition, \textcite{Weber2011, Zhu2014} noted similar inability to predict
the dependence of ignition delay on initial oxygen concentration for
\nBuOH{} for several different kinetic mechanisms.

\textcite{Karwat2011a} studied the ignition delays of \nBuOH{}
for stoichiometric mixtures over temperatures from \SIrange{920}{1040}{\kelvin}
and pressures near \SI{3}{\atmosphere}. \textcite{Karwat2011a} found good
agreement of the ignition delays with the kinetic model developed in the
study of \textcite{Black2010}. In addition, \textcite{Karwat2011a} used
a high-speed sampling valve to remove gas samples from the reaction chamber
during the induction period of \nBuOH{} ignition. They quantified
mole fractions of CH$_4$, CO, C$_2$H$_4$, C$_3$H$_6$, C$_2$H$_4$O, C$_4$H$_8$O,
1-C$_4$H$_8$, and \nBuOH{} at several times during the ignition.
Comparison of the time histories of these species with predictions from
the model by \textcite{Black2010} showed that, although the model is able
to predict the ignition delay well, it is not able to reproduce the time history
of species concentrations very well, particularly C$_2$H$_4$. This result
demonstrates the importance of rigorously validating a kinetic model over
a wide range of conditions and for a wide range of validation targets.

In comparison to the butanol isomers, \iPeOH{} has received significantly
less focus in the literature. Studies of the combustion of \iPeOH{}
have been conducted in JSRs \cite{Dayma2011,Togbe2011,Sarathy2013}, low-pressure flow
reactors \cite{Welz2012}, counterflow flame experiments \cite{Sarathy2013},
shock tubes \cite{Sarathy2013, Tsujimura2012, Tang2013}, and RCMs
\cite{Sarathy2013, Tsujimura2012}. Other studies have investigated
the efficacy of using \iPeOH{} in an HCCI engine \cite{Tsujimura2011,
Yacoub1998, Yang2010}. Finally, studies described in this work have
been conducted to determine the ignition properties of \iPeOH{} (see \cref{chap:peoh}).
Both the works by \textcite{Tsujimura2012, Sarathy2013} developed
detailed kinetic models for the combustion of \iPeOH{} whose validation
was based, in part, on ignition delay experiments. Using shock tubes and
RCMs in concert, these studies were able to provide ignition delays
for temperatures, pressures, and equivalence ratios of
\SIrange{650}{1450}{\kelvin}, \SIrange{7}{60}{\bar}, and $\phi =
\numlist{0.5;1.0;2.0}$, respectively. These studies generally found good
agreement of their models with their validation data sets, although
\textcite{Sarathy2013} found that their model had difficulty predicting
rich ignition delays. In addition, substantial pre-ignition heat release
was observed for all of the equivalence ratios at \SI{40}{\bar} in the
RCM measurements, similar to \tBuOH{}.

\section{Recent Work on Ignition of Methylcyclohexane}

Several studies have suggested the use of methylcyclohexane (MCH) as a
component in surrogate formulations \cite{Bieleveld2009,Naik2005}, as
discussed previously. Furthermore, MCH is the simplest branched or
substituted cycloalkane, and can therefore provide a base from which
to build models of the combustion of other, larger, naphthenes.

Substantial experimental and modeling work has been conducted for
napthenes in general, and MCH in particular. \textcite{Pitz2011} has
conducted an extensive review of the work on naphthenes, so only studies
involving homogeneous ignition of MCH are discussed here. Ignition delays of MCH
have been measured in shock tubes \cite{Rotavera2013, Vasu2009,
Vanderover2009, Hawthorn1966, Orme2006, Hong2011} and RCMs \cite{Tanaka2003,
Pitz2007, Mittal2009} by a number of researchers. These studies collectively
cover the temperature-pressure space in the range of \SIrange{700}{2100}{\kelvin}
and \SIrange{1}{70}{\atmosphere}. To complement this experimental work, a number
of kinetic models for MCH combustion have been constructed, notably by
\textcite{Orme2006, Pitz2007}.

The study of \textcite{Rotavera2013} measured ignition delays of MCH behind
reflected shock waves near \SIlist{1;10}{\atmosphere} for equivalence
ratios of $\phi = \numlist{0.5;1.0;2.0}$. They compared their measured ignition
delays with predictions from the model of \textcite{Pitz2007} and found
generally good agreement. \textcite{Hong2011} measured ignition delays
for conditions of temperature between \SIrange{1280}{1480}{\kelvin}, pressures
of \SIlist{1.5;3}{\atmosphere}, and equivalence ratios of $\phi = \numlist{1.0;0.5}$.
\textcite{Hong2011} compared their measurements with three mechanisms
from the literature, including those by \textcite{Pitz2007, Orme2006}
and found relatively good agreement for their conditions.

However, other studies have found that the existing models are not able
to predict ignition delays at conditions for which they were not validated%
---that is, the models are not truly predictive. For instance, previous work
conducted in an RCM by \textcite{Mittal2009} measured the ignition delays
of MCH/O$_2$/N$_2$/Ar mixtures at pressures of \SIlist{15.1;25.5}{\bar},
for three equivalence ratios of $\phi = \numlist{0.5;1.0;1.5}$, and over
the temperature range of \SIrange{680}{840}{\kelvin}. They compared
their measured ignition delays to simulated ignition delays computed
using the mechanism of \textcite{Pitz2007} and found that the model
over-predicted both the first stage and overall ignition delay substantially
\cite{Mittal2009}. Moreover, studies conducted in shock tubes by
\textcite{Vasu2009, Vanderover2009} came to similar conclusions in their
studies, which collectively considered conditions between \SIrange{795}{1560}{\kelvin}
and \SIrange{1}{70}{\atmosphere}. Further studies conducted in this work
have expanded the validation range of MCH ignition data and substantially
improved the predictive ability of kinetic models of MCH combustion (see
\cref{chap:mch}).

\section{Summary}

Due to its relevance in predicting the performance of a fuel in existing
and advanced engines, ignition delay is a very common measure of the
global performance of a kinetic mechanism. Ignition delays for homogeneous
systems are typically measured in shock tubes or RCMs, where the effects
of fluid motion and turbulence are generally minimized. However, as demonstrated
for the case of butanol isomers, \iPeOH{}, and methylcyclohexane, the
validation target of ignition delays is necessary but not sufficient to
develop truly predictive models.

Work has begun to utilize homogeneous ignition experimental
platforms to measure characteristics other than ignition delays---%
see, for instance, the work by \textcite{Karwat2011a} to measure species
profiles in their RCF using gas sampling; the work by \textcite{Das2012,Uddi2012}
to directly measure the temperature in the reaction chamber of an RCM using
mid-IR laser light absorption; and the work by \textcite{Stranic2013}
to perform simultaneous concentration measurements of multiple species in their
shock tube.

Moreover, fuels often exhibit substantially different ignition behavior
at engine-relevant, high-pressure, low- to intermediate-temperature
conditions. Therefore it is critically important to generate validation
datasets at these conditions.

Thus, the objectives of this work can be stated succinctly as follows:

\begin{enumerate}
\item \label{item:1} Generate ignition delay datasets for alternative
fuels at high-pressure, low-temperature conditions that have not been
studied extensively in previous work

\item \label{item:2} Develop and characterize a new experimental
apparatus to enable ex situ species measurements from the reaction
chamber of the rapid compression machine during the ignition delay

\item Use the data acquired from \cref{item:1} and \cref{item:2}
to extend the validation of new and existing chemical kinetic models
for the combustion of alternative fuels

\item Analyze new and existing chemical kinetic models to help
understand the cause of discrepancies and clarify the important
reaction pathways in high-pressure alternative fuel ignition
\end{enumerate}

\section{Organization of this Work}

The remainder of this work is structured as follows.
\Cref{chap:facilities} describes the experimental facilities
used in this work, including the rapid compression machine and the newly-developed
fast sampling system. Detailed uncertainty analyses are considered for the
appropriate apparatuses. The subsequent sections are organized by the fuel
studied: \cref{chap:buoh} considers the butanol isomers, \cref{chap:peoh}
considers \iPeOH{}, and \cref{chap:mch} considers methylcyclohexane.
Finally, \cref{chap:conclusions} presents conclusions based on this work and
recommendations for future directions.
\end{document}
