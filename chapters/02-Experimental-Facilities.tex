% arara: xelatex: { synctex: on, shell: off }
% arara: biber
% arara: xelatex: { synctex: on, shell: off }
% arara: sumatrapdf
\documentclass[12pt, letterpaper]{article}

%Set the document font
\usepackage{fontspec}
\setmainfont[Renderer=Basic,Ligatures=TeX]{Times New Roman}

%Set the size of the margins and the paper
\usepackage[margin=1in, letterpaper]{geometry}

\usepackage{mathtools}

%Set the color of the links and PDF metadata
\usepackage[
    colorlinks=true,
    citecolor=blue,
    linkcolor=black
]{hyperref}

\hypersetup{%
    pdfinfo={
        Title={High Pressure Ignition Chemistry of Alternative Fuels},
        Author={Bryan W. Weber}
    }
}
%Set up the page numbers
%This has to go after geometry so the page number is centered
\usepackage{fancyhdr}
\pagestyle{fancy}
\fancyhf{}
\fancyfoot[C]{\thepage}
\renewcommand{\headrulewidth}{0pt}

%Set a command to easily skip a line
\newcommand{\blankline}{\vspace*{\baselineskip}}

%Set up biblatex
\usepackage[
    backend=biber,
    url=false,
    doi=true,
    sorting=none,
    sortcites=true,
    maxbibnames=6,
    minbibnames=6,
    maxcitenames=2,
    mincitenames=1,
    citestyle=numeric-comp,
    firstinits=true,
    isbn=false
]{biblatex}
\addbibresource{../library.bib}

%Remove the "In:" from before the journal title for articles
\renewbibmacro{in:}{%
  \ifentrytype{article}{}{\printtext{\bibstring{in}\intitlepunct}}}

%Set the sort order of the names in each bibliography entry
\DeclareNameAlias{default}{last-first}

%Don't print the article title. To print the title, add #1 to the last {}
\DeclareFieldFormat[article,incollection,unpublished]{title}{}

%Add Vol. and No. before volume and issue.
\DeclareFieldFormat[article]{volume}{\bibstring{volume}\addspace #1}
\DeclareFieldFormat[article]{number}{\bibstring{number}\addspace #1}

%Put a comma between the volume and issue instead of period
\renewbibmacro*{volume+number+eid}{%
  \printfield{volume}%
  \setunit{\addcomma\space}%<---- was \setunit*{\adddot}%
  \printfield{number}%
  \setunit{\addcomma\space}%
  \printfield{eid}}

%Add a comma after the journal title
\renewbibmacro*{journal+issuetitle}{%
  \usebibmacro{journal}%
  \setunit*{\addcomma\addspace}%
  \iffieldundef{series}
    {}
    {\newunit
     \printfield{series}%
     \setunit{\addspace}}%
  \usebibmacro{volume+number+eid}%
  \setunit{\addspace}%
  \usebibmacro{issue+date}%
  \setunit{\addcolon\space}%
  \usebibmacro{issue}%
  \newunit}

%Set the text to double spacing
\usepackage[doublespacing]{setspace}

%Packages not present in main.tex preamble
\usepackage{booktabs}

\def\chapterautorefname~#1\null{Chap.~#1\null}
\def\sectionautorefname~#1\null{Sec.~#1\null}
\def\subsectionautorefname~#1\null{Sec.~#1\null}
\def\figureautorefname~#1\null{Fig.~#1\null}
\def\tableautorefname~#1\null{Table~#1\null}
\def\equationautorefname~#1\null{Eq.~(#1)\null}

\newcommand{\Autoref}[1]{%
  \begingroup%
  \def\chapterautorefname~##1\null{Chapter~##1\null}%
  \def\sectionautorefname~##1\null{Section~##1\null}%
  \def\subsectionautorefname~##1\null{Section~##1\null}%
  \def\figureautorefname~##1\null{Figure~##1\null}%
  \def\tableautorefname~##1\null{Table~##1\null}%
  \def\equationautorefname~##1\null{Equation~##1\null}%
  \autoref{#1}%
  \endgroup%
}

\usepackage[font={footnotesize}]{caption}

\defaultfontfeatures{Ligatures=TeX}

\begin{document}
\section{Rapid Compression Machine}
\label{sec:rcm}
\subsection{Experimental Procedure}
The studies in this dissertation were conducted using the Rapid
Compression Machine (RCM) constructed by Mittal around 2005 and described in the
work of \textcite{Mittal2007,Mittal2006a}. This RCM has been used to
study the autoignition behavior of a number of fuels, including
\textit{n}-decane, methylcyclohexane, hydrogen, syngas,
dimethyl ether, methanol, toluene, benzene, di-isobutylene, iso-octane,
jet fuel, and gasoline \cite{Kumar2009, Mittal2009, Das2012a, Mittal2006,
Das2012, Mittal2008a, Kumar2011a, Mittal2007a, Mittal2008, Kumar2010,
Dooley2010, Dooley2012, Hui2012a, Keromnes2013, Kukkadapu2013, Kukkadapu2012a}.

A modern RCM operates by rapidly compressing (hence the name) a test gas
mixture to targeted pressure and temperature conditions. The compression
is effected by either a single piston or dual, opposed pistons. Upon reaching
the targeted state, the piston is stopped and fixed in place so that the reactions
proceed in a constant volume reactor. When studying autoignition with an RCM,
the primary data are the measured pressure traces during and after the
compression stroke. These pressure traces are processed to derive
information such as the pressure and temperature at the end of
compression (EOC) and the ignition delay that are typically reported.
It is also possible to employ laser diagnostics or extract gas samples
from the reactor to examine reaction pathways in more detail.

The present RCM is a pneumatically-driven/hydraulically-stopped
single-piston arrangement. A schematic of the RCM is shown in
\autoref{fig:rcm-schematic}. The RCM consists of four chambers and
three pistons that are used to control machine. The chambers are
called the reaction chamber, the hydraulic chamber, the pneumatic
chamber, and the driving tank; similarly, the pistons are called
the reactor, hydraulic, and pneumatic pistons and are each installed
in the chamber of the same name. The rear of the reaction chamber
is bolted to the front of the hydraulic chamber; seals in the face
of the hydraulic chamber prevent oil from leaking and contaminating
the reaction chamber. The driving tank and the rear of the pneumatic
chamber are connected by a union; a seal around the circumference of
the pneumatic piston seals gas in the driving tank from the front of
the pneumatic chamber. Thus, the pneumatic piston can be driven by
pressure from the driving tank on its rear and pressure from the
pneumatic chamber on its front. The three pistons are connected by
a rod running from the front of the pneumatic piston to the rear of
the reactor piston so that they move as one; this will be referred
to as the piston assembly.

At the start of an experimental run, with the piston in the
EOC position, the reaction chamber is vacuumed to less
than one Torr. Next, the piston assembly is retracted pressurizing
the front face of the piston in the pneumatic chamber.
For safety, and to prevent damage to the RCM, the driving tank should
be filled to limit the acceleration of the piston assembly during this
retraction.
The pressure on the front of the pneumatic piston pulls the
piston assembly rearward and seats the rear of the
hydraulic piston onto an o-ring in the rear of the
hydraulic chamber. Then the hydraulic chamber is filled with oil to
a pressure of approximately 800 psi, providing a rearward force on the
front face of the hydraulic piston. The pressure is then released from
the front of the pneumatic chamber and the driving tank is filled to
the desired driving pressure. The
force on the hydraulic piston opposes the force on the pneumatic piston
from the driving tank and the piston assembly remains at rest. Then, the
reaction chamber is filled with the required initial pressure of test
gas mixture from the mixing tank. Finally, compression is triggered by
releasing the hydraulic pressure through an electrically operated solenoid
valve. The piston assembly is driven forward by the unbalanced force from
the pressure in the driving tank on the pneumatic piston
to compress the test mixture. The gases
in the reaction chamber are brought to the compressed pressure ($P_C$) and
compressed temperature ($T_C$) conditions in approximately 30-50
milliseconds.

The required driving pressure for a given EOC pressure can be estimated
from a force balance between the force on the pneumatic piston from the
driving tank and the force on the reactor piston from the reaction gases,
as shown in \autoref{eq:driving-pressure}.

\begin{subequations}
\label{eq:piston-force}
\begin{align}
    P_{d,\text{min}} \cdot A_p &= P_{r,\text{EOC}} \cdot A_r \\
    P_{d,\text{min}} \cdot \frac{\pi d_p^2}{4} &= P_{r,\text{EOC}} \cdot \frac{\pi d_r^2}{4} \\
    P_{d,\text{min}} &= P_{r,\text{EOC}} \cdot \frac{d_r^2}{d_p^2} \label{eq:driving-pressure}
\end{align}
\end{subequations}

In \autoref{eq:piston-force}, $P_{d,\text{min}}$ is the minimum
driving pressure, $A_p$ is the cross-sectional area of the pneumatic piston,
$P_{r,\text{EOC}}$ is the pressure in the reactor at the EOC (i.e. $P_C$),
$A_r$ is the cross-sectional area of the reactor piston, $d_p$ is the diameter
of the pneumatic piston, and $d_r$ is the diameter of the reactor piston.

The minimum driving pressure is such that the piston does not rebound at
the EOC due to pressure on the reactor piston. So that the driving
pressure can be much lower than the EOC pressure, the diameter ratio of
the reactor piston to the driver piston is $2/5$, allowing a factor of
6.25 lower driving pressure than EOC pressure. The actual driving
pressure should exceed the minimum by some safety margin so that the
reactor remains at constant volume even if there is some pressure rise
due to first stage ignition.

There is not a theoretical upper limit on the driving pressure. It is desired that the piston should
reach the EOC conditions in as short a time as possible to minimize heat
loss from the reactants to the reactor walls and minimize the time for
reactions to occur during the compression stroke. This implies that the
driving pressure should be made as high as possible so that the highest
piston velocity is achieved. However, higher piston velocities require
a higher deceleration at the EOC. In the present RCM, the deceleration
is provided by venting the hydraulic oil between steps on the hydraulic
piston and matched steps on the front of the hydraulic chamber. If the
piston is overdriven---that is, the driving pressure is too high---the
piston will not be sufficiently decelerated by the oil venting and will
impact the front of the hydraulic chamber at high velocity. This can damage
the RCM and cause the piston to rebound elastically. It also generates
substantial noise in the pressure trace and should be avoided.

Typical driving gas pressures are between 50 psi for $P_C = 15$ bar experiments
to $P_C = 125$ psi for 50 bar experiments. These driving pressures represent a
good compromise between the minimum required for no rebound at EOC due
to pressure and no rebound at EOC due to elastic reaction. Nonetheless,
a small amount of piston rebound can be expected during/after the
main ignition event. This small rebound may have an effect on the computation of
ignition delay if it reduces the pressure rise rate during the ignition;
it is expected that this effect will be very small relative to the
typical random uncertainty in ignition delay experiments. Moreover,
the driving pressures required to balance the full pressure rise during
to ignition are more likely cause elastic rebound,
especially for high $P_C$ when the post-ignition pressure rise is greater.

The EOC conditions ($P_C$ and $T_C$) can be independently varied. This
is made possible by independent variation of the compression ratio,
initial pressure and initial temperature, and the specific heat ratio
of the test gases. The compression ratio can be
increased by adding spacers onto the rear of the hydraulic chamber,
increasing the stroke, and can be reduced by adding split shims onto
the rear of the reaction chamber, increasing the EOC clearance length.

Fuel/oxidizer pre-mixtures are prepared in two mixing tanks, one approximately
17 L and the other approximately 15 L in volume. These large volumes allow many
runs to be conducted from one mixture preparation. The mixing tanks are connected
to the reaction chamber by flexible stainless steel manifold tubing. The tanks, reaction chamber,
and connecting manifold are wrapped in heating tape and insulation to control the initial
temperature of the mixture. Temperature controllers from Omega Engineering use thermocouples
placed on the lid of each mixing tank, approximately in the center of each mixing tank, embedded in
the wall of the reaction chamber, and near the inlet valve of the reaction chamber to control the
preheat temperature of the mixture. A static pressure transducer (Omega Engineering, 0-5200 Torr)
measures the pressure in the manifold and mixing tanks. This transducer is used
during mixture preparation and to measure the initial pressure of a given experiment.

Most of the fuels studied in this work are liquids at room temperature and
pressure and have relatively low vapor pressure. A similar procedure, outlined
below, was used for all of the butanol isomers, \textit{iso}-pentanol, and
methylcyclohexane. First mixing tanks are vacuumed to an ultimate pressure
less than 5 Torr. The liquid fuel is massed in a syringe to a precision of
0.01 g prior to injection through a septum. Proportions of O$_2$, N$_2$, and
Ar are added manometrically at room temperature. The preheat temperature of
the RCM is set above the saturation point for each fuel to ensure complete
vaporization. The vapor pressure as a function of temperature is calculated
according to fits taken from \textcite{Yaws1999}. A magnetic stirrer mixes
the reactants. The temperature inside the mixing tank is allowed to
equilibrate for approximately 1.5 hours.

This approach to mixture preparation has been validated in several previous
studies by withdrawing gas samples from the mixing tank and analyzing the
contents by GC/MS \cite{Weber2011}, GC-FID \cite{Kumar2009}, and GC-TCD
\cite{Das2012}. These studies have verified the concentration of
\textit{n}-butanol, \textit{n}-decane, and water, respectively. In addition,
both the work by \textcite{Kumar2009} on \textit{n}-decane and the study of
\textcite{Weber2011} on \textit{n}-butanol confirmed that there was no fuel
decomposition over the course of a typical set of experiments. Furthermore,
within this study, each new mixture preparation is checked against previously
tested conditions to ensure reproducibility.

The pressure in the reaction chamber during an experiment is monitored by a
Kistler 6125B piezoelectric dynamic pressure transducer. The charge signal from the
transducer is amplified and converted to a voltage by a Kistler 5010B charge amplifier.
The voltage is sent to a National Instruments cDAQ equipped with the NI-XXXX module.
The signal is recorded by LabView at 50 kHz.

\label{sec:ig-delay-def}
\subsection{Definition of Ignition Delay}
\Autoref{fig:ig-delay-def} shows a representative pressure trace from
these experiments with methylcyclohexane at $P_C=50$ bar, $T_C=761$ K,
and $\phi=1.5$ (See \autoref{sec:mch}). Note that \autoref{fig:ig-delay-def}
shows a case with two stages of ignition; not all of the fuels studied
had conditions that showed two-stage ignition. Nonetheless, the ignition
delay is consistently defined in all the work in this study. The
definitions of the EOC and the ignition delays are indicated on the figure.
The end of compression time is defined as the time when the pressure
reaches its maximum before first stage ignition occurs, or for cases
where there is no first stage ignition, the maximum pressure before
the overall ignition occurs. The first stage ignition delay is the time
from the end of compression until the first peak in the time derivative
of the pressure. The overall ignition delay is the time from the end of
compression until the largest peak in the time derivative of the pressure.

Each unique $P_C$ and $T_C$ condition is repeated at least 5 times to
ensure repeatability of the experiments. The experiment closest to the
mean of the runs at a particular condition is chosen for analysis and
presentation. The standard deviation of all of the runs at a condition
is less than 10\% of the mean in all cases.

\subsection{Non-Reactive Experiments}

\Autoref{fig:ig-delay-def} also shows a non-reactive pressure trace.
Due to heat loss from the test mixture to the cold reactor walls,
the pressure and temperature of the gas in the reaction chamber will
decrease after the end of compression. A non-reactive pressure trace
is measured that corresponds to each unique $P_C$ and $T_C$ condition
studied to quantify the effect of the heat loss on the ignition process
and to verify that no heat release has occurred during the compression
stroke. The non-reactive pressure trace is acquired by replacing the
oxygen in the oxidizer with nitrogen, so that the specific heat ratio
of the initial mixture is maintained, but the heat release due to
exothermic oxidation reactions is eliminated. Maintaining a similar
specific heat ratio ensures that the non-reactive experiment faithfully
reproduces the conditions of the reactive experiment. A representative
non-reactive pressure trace is shown in \ref{fig:ig-delay-def}
corresponding to the experimental conditions in the figure.

\subsection{New section}

An RCM to be used for studies of homogeneous chemistry---as in this study---%
must ensure that homogeneous conditions exist inside the reaction
chamber for the duration of the experiment. Due to the high piston
velocities required to minimize heat loss and reaction during the
compression stroke, complex fluid mechanical effects can strongly
affect the state of the reactants at the EOC. The most important of these
effects is caused by the motion of the piston itself, where the piston
pushes the wall boundary layer into a roll-up vortex \cite{Lee1998}.
This cold vortex mixes with the hotter gases near the center of
the reaction chamber and causes large spatial inhomogeneities of
temperature and species.

To facilitate spatially homogeneous conditions in the reactor
and reduce the effect of the roll-up vortex, it is necessary to trap
the boundary layer. This is accomplished on the present RCM by a
crevice machined into the crown of the piston, shown in cross-section
in \autoref{fig:piston}. The boundary layer enters the crevice through
the converging section as the piston moves forward and is trapped
within the crevice. The dimensions of the crevice were optimized
by \textcite{Mitall2006a} through CFD simulations for high-pressure
conditions. Subsequently, \textcite{Mittal2006b} experimentally showed that
the optimized crevice design provides homogeneous conditions in the
reaction chamber up to approximately 150 milliseconds after the EOC.
By using PLIF measurements of acetone-seeded mixtures,
\textcite{Mittal2006b} showed that there was a core region of gases
near the center of the reactor whose temperature remained spatially
homogeneous.

\subsection{Determination of Reactant Temperature}

Two independent thermodynamic properties are required to fix the
thermodynamic state of the reactants in the reaction chamber at a
given time. The first property is the pressure, measured by the dynamic
pressure transducer, as discussed previously; the second property
is chosen to be the temperature.

In general, it is rather difficult to directly measure the temperature
of the gases in the reaction chamber during and after compression.
Intrusive methods such as thermocouples may introduce inhomogeneities into
the reaction chamber and non-intrusive optical techniques are difficult to set up
and require extensive calibration at the pressures of interest in RCM
studies. Thus, the temperature is determined indirectly by applying an
assumption called the "adiabatic core hypothesis" to the reaction chamber
\cite{Mittal2007, Lee1998}.

If all of the gases in the reaction chamber are compressed isentropically,
the temperature at the end of compression can be found by the
following relations:

\begin{subequations}
\label{eq:tic}
\begin{align}
\ln\left(\text{CR}\right) = \int_{T_0}^{T_{ic}} \! \frac{1}{T\left(\gamma-1\right)} \, \mathrm{d} T \\
\ln\left(\frac{P_{ic}}{P_0}\right) = \int_{T_0}^{T_{ic}} \! \frac{\gamma}{T\left(\gamma-1\right)} \, \mathrm{d} T
\end{align}
\end{subequations}

\noindent
where CR is the volumetric compression ratio, $T_0$ is the initial temperature,
$T_{ic}$ is the temperature at the end of isentropic compression, $\gamma$ is the
temperature-dependent ratio of specific heats, $P_{ic}$ is the pressure at the
end of isentropic compression, and $P_0$ is the initial pressure.

However, experiments show that the measured pressure in the reaction chamber
does not reach the value of $P_{ic}$ calculated by using the geometric
compression ratio. The difference is due to finite heat loss from the
reactants to the reactor walls and the crevice volume during the
compression. Under the adiabatic core hypothesis, it is assumed that
the heat loss from the reactants only occurs in a thin boundary layer
near the wall, and the central core region is unaffected by heat loss
(i.e. it is the core is adiabatic) \cite{Desgroux1995}. Thus, the heat
loss is modeled as an effective reduction in the compression ratio, and
the temperature during the compression stroke can be calculated by:

\begin{align}
\ln\left(\frac{P_{C}}{P_0}\right) = \int_{T_0}^{T_{C}} \! \frac{\gamma}{T\left(\gamma-1\right)} \, \mathrm{d} T
\label{eq:tc}
\end{align}

\noindent
where $P_C$ is the measured pressure at the end of compression, $T_C$
is the temperature at the end of compression, and the other variables
are the same as in \autoref{eq:tic}.

After the end of compression, the pressure in the reaction chamber
decreases, as can be seen in \autoref{fig:ig-delay-def}. This pressure
decrease is caused by heat loss from the constant volume reaction
chamber and is accompanied by a decrease in the temperature of the
reactants. To model the thermodynamic state after the end of compression,
the adiabatic core hypothesis is again applied, and the heat loss is
assumed to occur only in a thin boundary layer near the reactor walls.
Thus, the core region is again modeled as adiabatic, and the heat loss
from the boundary layer can be modeled as an isentropic volume
expansion.

\subsection{Determination of Compressed Temperature}

In general, the specific heat ratio is a function of temperature, so
\autoref{eq:tc} cannot be integrated directly. If the specific heats
are parameterized with a polynomial fit, it is possible to integrate
\autoref{eq:tc} directly, but this process is quite tedious; nonetheless,
it will be applied in \autoref{sec:uncertainty} to determine the uncertainty
of $T_C$. In general, the simplest method of calculating $T_C$ is to use
software numerically integrate \autoref{eq:tc}.

In this work, the CHEMKIN-Pro \cite{Chemkin2012} software is used to
perform the numerical integration and calculation of $T_C$. The
CHEMKIN-Pro software provides the facility for a user-specified
volume as a function of time to be applied to a homogeneous,
adiabatic reactor. Since the adiabatic core of the reaction chamber
is modeled as undergoing an isentropic volumetric compression followed
by an isentropic volumetric expansion, the user-specified volume
functionality is used to compute the RCM reactor state as a function
of time. A volume trace for simulation is computed from the measured
pressure trace using the isentropic relation:

\begin{align}
\frac{V_2}{V_1} = \left[\frac{P_1}{P_2}\right]^{\frac{1}{\gamma}}
\label{eq:volume-trace}
\end{align}

\noindent
where $V_1$ and $V_2$ are the volumes at consecutive time points,
$P_1$ and $P_2$ are the pressures at consecutive time points, and
$\gamma$ is the temperature dependent specific heat. This equation
is applied during and after the compression stroke to calculate
the volume trace. 

For use in \autoref{eq:volume-trace}, $\gamma$ is tabulated for each
time point. Thus, the temperature at each time point must also be
computed by using the isentropic relation for temperature:

\begin{align}
\frac{T_2}{T_1} = \left[\frac{P_2}{P_1}\right]^{\frac{\gamma-1}{\gamma}}
\label{eq:isen-temp}
\end{align}

\noindent
where $T_2$ and $T_1$ are the temperature at consecutive time points.
Since $T_2$ depends on the value of $\gamma$, which in turn depends
on $T_2$, \autoref{eq:isen-temp} is iterated until the temperature
changes by less than one tenth of one percent on consecutive iterations.
The temperature calculated by \autoref{eq:isen-temp} is typically within
1K of the temperature calculated by CHEMKIN-Pro; the difference is due
to differences in the time step used by CHEMKIN-Pro.


\subsection{Uncertainty of Ignition Delay and Compressed Temperature}
\label{sec:uncertainty}


\end{document}