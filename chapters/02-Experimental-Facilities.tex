% arara: xelatex: { synctex: on, shell: off }
% arara: biber
% arara: xelatex: { synctex: on, shell: off }
% arara: sumatrapdf
\documentclass[12pt, letterpaper]{article}

%Set the document font
\usepackage{fontspec}
\setromanfont{Times New Roman}

%Set the size of the margins and the paper
\usepackage[margin=1in, letterpaper]{geometry}

%Set the color of the links and PDF metadata
\usepackage[
    colorlinks=true,
    citecolor=blue,
    linkcolor=black
]{hyperref}

\hypersetup{%
    pdfinfo={
        Title={High Pressure Ignition Chemistry of Alternative Fuels}
        Author={Bryan W. Weber}
    }
}
%Set up the page numbers
%This has to go after geometry so the page number is centered
\usepackage{fancyhdr}
\pagestyle{fancy}
\fancyhf{}
\fancyfoot[C]{\thepage}
\renewcommand{\headrulewidth}{0pt}

%Set a command to easily skip a line
\newcommand{\blankline}{\vspace*{\baselineskip}}

%Set up biblatex
\usepackage[
    backend=biber,
    url=false,
    doi=true,
    sorting=none,
    sortcites=true,
    maxbibnames=6,
    minbibnames=6,
    maxcitenames=2,
    mincitenames=1,
    citestyle=numeric-comp,
    firstinits=true,
    isbn=false
]{biblatex}
\addbibresource{../library.bib}

%Remove the "In:" from before the journal title for articles
\renewbibmacro{in:}{%
  \ifentrytype{article}{}{\printtext{\bibstring{in}\intitlepunct}}}

%Set the sort order of the names in each bibliography entry
\DeclareNameAlias{default}{last-first}

%Don't print the article title. To print the title, add #1 to the last {}
\DeclareFieldFormat[article,incollection,unpublished]{title}{}

%Add Vol. and No. before volume and issue.
\DeclareFieldFormat[article]{volume}{\bibstring{volume}\addspace #1}
\DeclareFieldFormat[article]{number}{\bibstring{number}\addspace #1}

%Put a comma between the volume and issue instead of period
\renewbibmacro*{volume+number+eid}{%
  \printfield{volume}%
  \setunit{\addcomma\space}%<---- was \setunit*{\adddot}%
  \printfield{number}%
  \setunit{\addcomma\space}%
  \printfield{eid}}

%Add a comma after the journal title
\renewbibmacro*{journal+issuetitle}{%
  \usebibmacro{journal}%
  \setunit*{\addcomma\addspace}%
  \iffieldundef{series}
    {}
    {\newunit
     \printfield{series}%
     \setunit{\addspace}}%
  \usebibmacro{volume+number+eid}%
  \setunit{\addspace}%
  \usebibmacro{issue+date}%
  \setunit{\addcolon\space}%
  \usebibmacro{issue}%
  \newunit}

%Set the text to double spacing
\usepackage[doublespacing]{setspace}

%Packages not present in main.tex preamble
\usepackage{booktabs}

\def\chapterautorefname~#1\null{Chap.~#1\null}
\def\sectionautorefname~#1\null{Sec.~#1\null}
\def\subsectionautorefname~#1\null{Sec.~#1\null}
\def\figureautorefname~#1\null{Fig.~#1\null}
\def\tableautorefname~#1\null{Table~#1\null}
\def\equationautorefname~#1\null{Eq.~(#1)\null}

\newcommand{\Autoref}[1]{%
  \begingroup%
  \def\chapterautorefname~##1\null{Chapter~##1\null}%
  \def\sectionautorefname~##1\null{Section~##1\null}%
  \def\subsectionautorefname~##1\null{Sub--Section~##1\null}%
  \def\figureautorefname~##1\null{Figure~##1\null}%
  \def\tableautorefname~##1\null{Table~##1\null}%
  \def\equationautorefname~##1\null{Equation~##1\null}%
  \autoref{#1}%
  \endgroup%
}

\usepackage[font={footnotesize}]{caption}

\usepackage{mathtools}

\begin{document}
The studies in this dissertation were conducted using the Rapid
Compression Machine (RCM) constructed by Mittal and described in the
work of \textcite{Mittal2007}.

The present RCM is a pneumatically-driven/hydraulically-stopped
single-piston arrangement. A schematic of the RCM is shown in
\autoref{fig:rcm-schematic}. The RCM consists of four chambers and
three pistons that are used to control machine. The chambers are
called the reaction chamber, the hydraulic chamber, the pneumatic
chamber, and the driving tank; similarly, the pistons are called
the reactor, hydraulic, and pneumatic pistons and are each installed
in the chamber of the same name. The driving tank and pneumatic
chamber are connected by a union and the pneumatic piston is sealed
to the walls of the pneumatic chamber, so that the pneumatic piston
is driven by pressure from the driving tank on its rear and pressure
from the pneumatic chamber on its front. The pistons are connected by
a rod running from the front of the pneumatic piston to the rear of
the reactor piston so that they move as one; this will be referred
to as the piston assembly.

At the start of an experimental run, with the piston in the
end-of-compression (EOC) position, the reaction chamber is vacuumed to less
than one Torr. Next, the piston assembly is retracted by pneumatic
pressure on the front face of the piston in the pneumatic chamber.
For safety, and to prevent damage to the RCM, the driving tank should
be filled to limit the acceleration of the piston assembly during this
retraction.
The pneumatic pressure on the front of the pneumatic piston pulls the
piston assembly rearward and seats the
hydraulic piston onto an o-ring in the rear of the
hydraulic chamber. Then the hydraulic chamber is filled with oil to
a pressure of approximately 800 psi, providing a rearward force on the
hydraulic piston. The pneumatic pressure is then released from the pneumatic
chamber and the driving tank is filled to the driving pressure. The
force on the hydraulic piston opposes the force on the pneumatic piston
from the driving tank and the piston assembly remains at rest. Then, the
reaction chamber is filled with the required initial pressure of test
gas mixture from the mixing tank. The compression is triggered by
releasing the hydraulic pressure. The piston assembly is driven forward
to compress the test mixture by high-pressure nitrogen gas in the driving
tank. The gases
in the test section are brought to the compressed pressure ($P_C$) and
compressed temperature ($T_C$) conditions in approximately 30
milliseconds.

The required driving pressure for a given EOC pressure can be estimated
from a force balance between the force on the pneumatic piston from the
driving tank and the force on the reactor piston from the reaction gases,
as shown in \autoref{eq:driving-pressure}.

\begin{subequations}
\label{eq:piston-force}
\begin{align}
    P_{d,\text{min}} \cdot A_p &= P_{r,\text{EOC}} \cdot A_r \\
    P_{d,\text{min}} \cdot \frac{\pi d_p^2}{4} &= P_{r,\text{EOC}} \cdot \frac{\pi d_r^2}{4} \\
    P_{d,\text{min}} &= P_{r,\text{EOC}} \cdot \frac{d_r^2}{d_p^2} \label{eq:driving-pressure}
\end{align}
\end{subequations}

In \autoref{eq:piston-force}, $P_{d,\text{min}}$ is the minimum
driving pressure, $A_p$ is the cross-sectional area of the pneumatic piston,
$P_{r,\text{EOC}}$ is the pressure in the reactor at the EOC (i.e. $P_C$),
$A_r$ is the cross-sectional area of the reactor piston, $d_p$ is the diameter
of the pneumatic piston, and $d_r$ is the diameter of the reactor piston.

The minimum driving pressure is such that the piston does not rebound at
the EOC due to pressure on the reactor piston. So that the driving
pressure can be much lower than the EOC pressure, the diameter ratio of
the reactor piston to the driver piston is $2/5$. The actual driving
pressure should exceed the minimum by some safety margin so that the
reactor remains at constant volume even if there is some pressure rise
due to first stage ignition.

There is not a theoretical upper limit on the driving pressure. It is desired that the piston should
reach the EOC conditions in as short a time as possible to minimize heat
loss from the reactants to the reactor walls and minimize the time for
reactions to occur during the compression stroke. This implies that the
driving pressure should be made as high as possible so that the highest
piston velocity is achieved. However, higher piston velocities require
a higher deceleration at the EOC. In the present RCM, the deceleration
is provided by venting the hydraulic oil between steps on the hydraulic
piston and matched steps on the front of the hydraulic chamber. If the
piston is "overdriven" - that is, the driving pressure is too high - the
piston will not be sufficiently decelerated by the oil venting and will
impact the front of the hydraulic chamber at high velocity. This can damage
the RCM and cause the piston to rebound elastically. It also generates
substantial noise in the pressure trace and should be avoided.

Typical driving gas pressures are between 50 psi for 15 bar experiments
to 125 psi for 50 bar experiments. These driving pressures represent a
good compromise between the minimum required for no rebound at EOC due
to pressure and no rebound at EOC due to elastic reaction. Nonetheless,
a small amount of piston rebound can be expected during/after the
main ignition event as the driving pressures required to overcome the
full pressure rise due to ignition would cause elastic rebound. Thus,
this small rebound may have an effect on the computation of
ignition delay if it reduces the pressure rise rate during the ignition;
it is expected that this effect will be very small relative to the
typical random uncertainty in ignition delay experiments.

\end{document}