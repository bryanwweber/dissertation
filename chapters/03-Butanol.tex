% arara: xelatex: { synctex: on, shell: off }
% arara: biber
% arara: xelatex: { synctex: on, shell: off }
% arara: sumatrapdf
\documentclass[12pt, letterpaper]{article}

%Set the document font
\usepackage{fontspec}
\setromanfont{Times New Roman}

%Set the size of the margins and the paper
\usepackage[margin=1in, letterpaper]{geometry}

%Set the color of the links and PDF metadata
\usepackage[
    colorlinks=true,
    citecolor=blue,
    linkcolor=black
]{hyperref}

\hypersetup{%
    pdfinfo={
        Title={High Pressure Ignition Chemistry of Alternative Fuels}
        Author={Bryan W. Weber}
    }
}
%Set up the page numbers
%This has to go after geometry so the page number is centered
\usepackage{fancyhdr}
\pagestyle{fancy}
\fancyhf{}
\fancyfoot[C]{\thepage}
\renewcommand{\headrulewidth}{0pt}

%Set a command to easily skip a line
\newcommand{\blankline}{\vspace*{\baselineskip}}

%Set up biblatex
\usepackage[
    backend=biber,
    url=false,
    doi=true,
    sorting=none,
    sortcites=true,
    maxbibnames=6,
    minbibnames=6,
    maxcitenames=2,
    mincitenames=1,
    citestyle=numeric-comp,
    firstinits=true,
    isbn=false
]{biblatex}
\addbibresource{../library.bib}

%Remove the "In:" from before the journal title for articles
\renewbibmacro{in:}{%
  \ifentrytype{article}{}{\printtext{\bibstring{in}\intitlepunct}}}

%Set the sort order of the names in each bibliography entry
\DeclareNameAlias{default}{last-first}

%Don't print the article title. To print the title, add #1 to the last {}
\DeclareFieldFormat[article,incollection,unpublished]{title}{}

%Add Vol. and No. before volume and issue.
\DeclareFieldFormat[article]{volume}{\bibstring{volume}\addspace #1}
\DeclareFieldFormat[article]{number}{\bibstring{number}\addspace #1}

%Put a comma between the volume and issue instead of period
\renewbibmacro*{volume+number+eid}{%
  \printfield{volume}%
  \setunit{\addcomma\space}%<---- was \setunit*{\adddot}%
  \printfield{number}%
  \setunit{\addcomma\space}%
  \printfield{eid}}

%Add a comma after the journal title
\renewbibmacro*{journal+issuetitle}{%
  \usebibmacro{journal}%
  \setunit*{\addcomma\addspace}%
  \iffieldundef{series}
    {}
    {\newunit
     \printfield{series}%
     \setunit{\addspace}}%
  \usebibmacro{volume+number+eid}%
  \setunit{\addspace}%
  \usebibmacro{issue+date}%
  \setunit{\addcolon\space}%
  \usebibmacro{issue}%
  \newunit}

%Set the text to double spacing
\usepackage[doublespacing]{setspace}

%Packages not present in main.tex preamble
\usepackage{booktabs}

\def\chapterautorefname~#1\null{Chap.~#1\null}
\def\sectionautorefname~#1\null{Sec.~#1\null}
\def\subsectionautorefname~#1\null{Sec.~#1\null}
\def\figureautorefname~#1\null{Fig.~#1\null}
\def\tableautorefname~#1\null{Table~#1\null}
\def\equationautorefname~#1\null{Eq.~(#1)\null}

\newcommand{\Autoref}[1]{%
  \begingroup%
  \def\chapterautorefname~##1\null{Chapter~##1\null}%
  \def\sectionautorefname~##1\null{Section~##1\null}%
  \def\subsectionautorefname~##1\null{Sub--Section~##1\null}%
  \def\figureautorefname~##1\null{Figure~##1\null}%
  \def\tableautorefname~##1\null{Table~##1\null}%
  \def\equationautorefname~##1\null{Equation~##1\null}%
  \autoref{#1}%
  \endgroup%
}

\usepackage[font={footnotesize}]{caption}

\usepackage{mathtools}

\usepackage{multirow}

\graphicspath{ {../figures/} }

\newcommand{\linebreakcell}[2][c]{%
  \begin{tabular}[#1]{@{}c@{}}#2\end{tabular}}

%End of extra imports

\begin{document}
\section{Experimental Procedure}
\label{sec:buoh-proc}

The reactants used in this study, along with their purities, are shown in
\autoref{tab:buoh-expts}. To determine the relative proportions of each
reactant in the mixture, the absolute mass of fuel, the equivalence ratio
($\phi$), and the oxidizer ratio ($X_{O_2}:X_{\mathrm{inert}}$, where $X$
indicates mole fraction) are specified. \textit{s}- and \textit{i}-Butanol are
liquid at room temperature and have relatively low vapor pressure; therefore,
each is measured gravimetrically in a syringe to within 0.01 g of the specified
value. \textit{t}-Butanol is solid at room temperature (melting point: $25^{\circ}$ C),
and is melted before being handled in the same procedure as the other fuels.
The 17 L mixing tank is vacuumed to an ultimate pressure less than 5 Torr prior
to the injection of the liquid fuel through a septum. Proportions of O$_2$ and
N$_2$ are added manometrically at room temperature. The preheat temperature of
the RCM is set above the saturation point for each fuel to ensure complete
vaporization. A magnetic stirrer mixes the reactants. The temperature inside
the mixing tank is allowed to equilibrate for approximately 1.5 hours.

This approach to mixture preparation has been validated in several previous
studies by withdrawing gas samples from the mixing tank and analyzing the
contents by GC/MS, GC-FID, and GC-TCD.16,29,30 These studies have verified the
concentration of \textit{n}-butanol, water, and \textit{n}-decane, respectively. In addition,
both the work by \textcite{Kumar2009} on \textit{n}-decane and the study of
\textcite{Weber2011} on \textit{n}-butanol confirmed that there was no fuel
decomposition over the course of a typical set of experiments. Furthermore,
within this study, each new mixture preparation is checked against previously
tested conditions to ensure reproducibility.

\Autoref{tab:buoh-expts} shows the experimental conditions considered in this
study. The compressed pressure conditions have been chosen to match the
previous \textit{n}-butanol study \cite{Weber2011}, but also to provide data in
regions not covered extensively in previous work. In addition, the fuel loading
conditions have been chosen to complement previous work; the studies by
\textcite{Stranic2012} and \textcite{Moss2008} used relatively dilute mixtures,
so we have included higher fuel loading conditions. Furthermore, the compressed
temperature conditions we have studied ($T_C=715-910$ K) have not been examined
in any other study, to our knowledge.

Each compressed pressure and temperature condition is repeated at least six
times to ensure repeatability. The mean and standard deviation of the ignition
delay for all runs at each condition are calculated. As an indication of
repeatability, the standard deviation is less than 10\% of the mean in every
case. Representative experimental pressure traces for simulations and
presentation are then chosen as the closest to the mean.

\begin{table}
    \centering
    \caption{Experimental Conditions and Reactant Purities}
    \label{tab:buoh-expts}
    \begin{tabular}{*{7}{c}}
    \toprule
    \multicolumn{5}{c}{Reactant (Purity)} & \multirow{3}[0]{*}{\linebreakcell{Equivalence \\ Ratio \\ $\phi$}} & \multirow{3}[0]{*}{\linebreakcell{Compressed \\ Pressure \\ $P_C$ (bar)}} \\
    \cmidrule{1-5}
    \linebreakcell{\textit{s}-butanol \\ (99.99\%)} & \linebreakcell{\textit{i}-butanol \\ (99.99\%)} & \linebreakcell{\textit{t}-butanol \\ (99.99\%)} & \linebreakcell{O$_2$ \\ (99.999\%)} & \linebreakcell{N$_2$ \\ (99.995\%)} & & \\
    \cmidrule{1-5}
    \multicolumn{5}{c}{Mole Percentage}   & & \\
    \midrule
    3.38  &       &       & 20.30 & 76.32 & 1.0   & 15 \\
    3.38  &       &       & 20.30 & 76.32 & 1.0   & 30 \\
          & 3.38  &       & 20.30 & 76.32 & 1.0   & 15 \\
          & 3.38  &       & 20.30 & 76.32 & 1.0   & 30 \\
          &       & 3.38  & 20.30 & 76.32 & 1.0   & 15 \\
          &       & 3.38  & 20.30 & 76.32 & 1.0   & 30 \\
          &       & 1.72  & 20.65 & 77.63 & 0.5   & 30 \\
          &       & 6.54  & 19.63 & 73.83 & 2.0   & 30 \\
    \bottomrule
    \end{tabular}
\end{table}

\section{Experimental Results}
\label{sec:buoh-expts}

\Autoref{fig:buoh-15bar} shows the ignition delays of the four isomers of
butanol measured in the RCM, at compressed pressure of $P_C=15$ bar for
stoichiometric mixture in air. The dashed line for each isomer is a least
squares fit to the data. The vertical error bars are two standard deviations
of the measurements of the ignition delay. The standard deviation is computed
based on all the runs at a particular compressed temperature and pressure
condition. A conservative estimate of the uncertainty in $T_C$ was calculated
in our previous work to be approximately 0.7–1.7\%. Due to the similar nature
of these experiments, and the similar properties of the fuels, this estimate
is considered to be valid for this study as well.

\Autoref{fig:buoh-15bar} demonstrates the differences in reactivity between
the isomers for stoichiometric fuel/air mixtures at compressed pressure
$P_C=15$ bar. \textit{n}-Butanol is clearly the most reactive, followed by
\textit{s}- and \textit{i}-butanol, which have very similar reactivities in
this temperature and pressure range. \textit{t}-Butanol is the least reactive.

The order of reactivity found in the RCM at 15 bar agrees with the shock tube
study at higher temperatures (approximately 1275−1667 K) and lower pressure
(1.5 atm) by \textcite{Stranic2012} but differs slightly from the studies of
\textcite{Moss2008} who measured ignition delays in a shock tube near 1.5 atm
and between 1275−1400 K, and \textcite{Veloo2011a} who measured
atmospheric-pressure laminar flame speeds. In particular, \textcite{Moss2008}
and \textcite{Veloo2011a} found distinct differences in reactivity between
\textit{s}- and \textit{i}-butanol, but the present study and the study by
\textcite{Stranic2012} found that they were nearly indistinguishable in terms
of reactivity under the conditions investigated. In addition,
\textcite{Stranic2012} noted some disagreement between their shock tube
ignition data and the data of \textcite{Moss2008} but their attempts to isolate
the cause could not discern what the difference might be caused by.

Further, the order of the reactivity of the butanol isomers also shows complex
temperature and pressure dependence. This is corroborated by the results shown
in \autoref{fig:buoh-30bar}. In \autoref{fig:buoh-30bar}, the order of
reactivity is different than in \autoref{fig:buoh-15bar}, where the only
variation between the plots is the compressed pressure; in
\autoref{fig:buoh-30bar} the compressed pressure $is P_C=30$ bar.
\autoref{fig:buoh-30bar} shows \textit{i}-butanol to be the least reactive,
\textit{s}-butanol to be less reactive than \textit{t}-butanol (but similar),
and \textit{n}-butanol to be the most reactive. Interestingly, the results of
the shock tube study by \textcite{Stranic2012} differ from those in the current
study at higher pressure (despite the agreement at lower pressure). In their
study, \textcite{Stranic2012} found \textit{i}- and \textit{n}-butanol to have
similar reactivity near 43 atm. in the temperature range of 1020–1280 K,
whereas in the present study we find \textit{i}-butanol to be the least
reactive of all four isomers at a pressure of 30 bar and over the temperature
range (715–910 K) investigated.

The fact that \textit{t}-butanol becomes relatively more reactive than
\textit{i}- and \textit{s}-butanol as pressure increases is surprising at first
glance, and the reasons are not immediately apparent. Closer examination of the
pressure traces for each experiment gives one clue as to the cause of the
increased reactivity. \Autoref{fig:tbuoh-15bar} shows the pressure traces for
the \textit{t}-butanol experiments at 15 bar for stoichiometric mixtures in
air. It is evident that there is some pre-ignition heat release, because the
reactive pressure trace diverges from the non-reactive case prior to the
ignition event. Of the other isomers of butanol, only \textit{n}-butanol shows
any visible heat release prior to the main ignition event at 15 bar.

\Autoref{fig:tbuoh-phi10} shows the pressure traces for \textit{t}-butanol
experiments at 30 bar for stoichiometric mixtures in air. The effect of
pre-ignition heat release is even more striking in this figure, with
substantial changes in the slope of the pressure trace during the reactive
runs. Comparing to the pressure traces of the other isomers once again shows
that the magnitude of the pre-ignition heat release for \textit{t}-butanol is
much greater. Despite the appearance of early pressure rise, which is typically
indicative of two-stage ignition and low temperature chain branching, we do not
find a negative temperature coefficient region in terms of the ignition delay
response for any \textit{t}-butanol experiments. Therefore, we adopt the phrase
"pre-ignition heat release" rather than “two-stage ignition” in this work.

In an effort to understand the reactions causing the pre-ignition heat release,
further experiments are conducted for \textit{t}-butanol at $P_C=30$ bar, for
equivalence ratios of 0.5 and 2.0 in air. \Autoref{fig:tbuoh-delays} shows
Arrhenius plots of the ignition delays for the three equivalence ratios. As
with the previous \textit{n}-butanol experiments at 15 bar \cite{Weber2011}
$\phi=0.5$ is the least reactive and $\phi=2.0$ is the most reactive. The
slopes are similar, indicating that the overall activation energies are similar
for the conditions investigated.

A more interesting comparison is of the pressure traces of the three
equivalence ratios. It is clear from Figures \ref{fig:tbuoh-phi10},
\ref{fig:tbuoh-phi05}, and \ref{fig:tbuoh-phi20} that there are qualitative
differences in the pre-ignition heat release between the three equivalence
ratios. This is most likely due to the effect of the increased (reduced) fuel
mole fraction in the $\phi=2.0$ ($\phi=0.5$) case, since the mole fraction of
fuel is changed by +93\% (-49\%) compared to the $\phi=1.0$ case, while the
mole fraction of oxygen changes by only -3\% (+2\%) compared to the $\phi=1.0$
case, as shown in \Autoref{tab:buoh-expts}. Therefore, it appears that the
qualitative change in pre-ignition behavior is due to the change of fuel mole
fraction, where higher fuel loading promotes pre-ignition heat release.

\end{document}