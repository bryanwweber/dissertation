% arara: lualatex: { synctex: on, shell: off }
% arara: biber
% arara: lualatex: { synctex: on, shell: off }
% arara: sumatrapdf
\documentclass[../main.tex]{subfiles}

\begin{document}
The detailed conclusions relevant to each of the experimental works
considered in this study are presented in their respective chapters.
The following gives a general summary of the previous works and provides
recommendations for future work, including descriptions of ongoing
investigations using a new sampling system.

\section{Conclusions}
\label{sec:overall-conclusions}
The studies reported in this work are the first experiments exploring
the low-to-intermediate temperature autoignition of the butanol isomers.
These data provide a unique look into the behavior of these fuels under
engine-relevant conditions. For the stoichiometric condition at two
pressures, \nBuOH{} is the most reactive of the isomers. However, the
order of the reactivity of the other isomers depends on the prevailing
pressure conditions during the induction period. \tBuOH{} becomes the
second most reactive isomer at the higher pressure condition and shows
unique behavior during the induction period. Analysis of a detailed
kinetic model for combustion of the butanol isomers is conducted to elucidate
the controlling chemistry during the autoignition of the four isomers,
and this analysis indicates that the different behavior of \tBuOH{} is
due to a unique set of controlling reactions for \tBuOH{}.

New experimental autoignition data collected for \iBuOH{} are used to
compare the important pathways of butanol combustion predicted by two
recent chemical kinetic mechanisms. The reactivity of each mechanism is
controlled by a different radical (hydroxyl vs.\ hydroperoxyl) because the
main fuel reaction pathways are also different. However, neither model
is able to predict properly the dependence of the ignition delay on initial
oxygen concentration. Overall, the importance of peroxy chemistry is
highlighted in this work and further computational and experimental
studies are needed to better understand the role of peroxy species in
the autoignition of alcohols.

An existing model for the combustion of \iPeOH{} is updated with newly
calculated rate coefficient estimates and newly discovered reaction
pathways. The model is compared to new and existing experimental data
from RCMs and STs and predicts the high-temperature ignition delays
fairly well. In addition, the model qualitatively predicts the slow
pressure rise noted during the induction period of low-temperature
autoignition. However, the model is not able to predict quantitatively
the ignition delay for off-stoichiometric mixtures of \iPeOH{} and air
at low temperatures.

Finally, new experimental data is collected for MCH at compressed
pressure of $P_C=\SI{50}{\bar}$. These data at three equivalence ratios
showed that the lean case is the most reactive and the rich case is the
least reactive (in terms of the inverse of ignition delay) because the
equivalence ratio was changed by varying the initial oxygen concentration
at constant initial fuel concentration. In addition, the data include the
characteristic NTC region for the rich and stoichiometric case, but the
ignition delay was too short to resolve the NTC for the lean case. An
existing model for MCH combustion was updated with new reaction rate
coefficient estimates and new reaction pathways. The new model shows
good agreement with the overall ignition delays of several datasets
including the new experimental data collected in this work. However,
the first stage ignition delay is uniformly under predicted. Pathway
and sensitivity analysis are used to identify the important reactions in
the model, including reactions of the primary fuel radicals and the peroxy
radicals formed from the primary fuel radicals.

\section{Future Work}
\label{sec:future-work}


\end{document}
