% arara: lualatex: { synctex: on, shell: off }
% arara: biber
% arara: lualatex: { synctex: on, shell: off }
% arara: sumatrapdf
\documentclass[../main.tex]{subfiles}

% %Set the package to import preambles
\usepackage{subfiles}

%Load graphicx here to specify options
\usepackage[final]{graphicx}

%Various math improvements.
%Must be loaded before hyperref
\usepackage{mathtools}

%Set the document font
%Set the math font. Has to come after mathtools because
%some font stuff gets overwritten.

% Uncomment the following lines to use Times New Roman for text and numbers
% and Cambria Math for math fonts
% \usepackage{unicode-math}
% \unimathsetup{math-style=TeX}
% \setmathfont[range=\mathup/{num}]{Times New Roman}
% \setmathfont[range=\mathit/{greek,Greek,latin,Latin}]{Cambria Math}
% \setmathfont[range=\mathup/{greek,Greek,latin,Latin}]{Cambria Math}
% \setmathfont[range={"2212,"002B,"003D,"0028,"0029,"005B,"005D,"221A,
% "2211,"2248,"222B,"007C,"2026,"2202,"00D7,"0302,"2261,"0025,"22C5,
% "00B1,"2194,"21D4,"2260}]
% {Cambria Math}
% \setmainfont[Ligatures=TeX]{Times New Roman}

% Uncomment the following lines to use TeX Gyre Termes for text and math
% fonts. This is very similar to Times and was used for the "official"
% versions of the thesis.
% \usepackage{unicode-math}
% \unimathsetup{math-style=TeX}
% \setmainfont{TeX Gyre Termes}
% \setmathfont{TeX Gyre Termes Math}


% Always use Inconsolata for the monospaced font.
\setmonofont{Inconsolata}

%Better looking fonts
\usepackage[final]{microtype}

%Set the text to double spacing
%According to hyperref README,
%setspace should be loaded first
\usepackage[doublespacing]{setspace}

%Set a command to easily skip a line
\newcommand{\blankline}{\vspace*{\baselineskip}}

%Set up biblatex
\usepackage[
    backend=biber,
    % url=false,
    doi=true,
    sorting=none,
    sortcites=true,
    maxbibnames=6,
    minbibnames=6,
    maxcitenames=2,
    mincitenames=1,
    citestyle=numeric-comp,
    firstinits=true,
    isbn=false
]{biblatex}
\addbibresource{/home/bryan/dissertation/library.bib}

%Remove the "In:" from before the journal title for articles
\renewbibmacro{in:}{%
  \ifentrytype{article}{}{\printtext{\bibstring{in}\intitlepunct}}}

%Change the name of the bibliography section to "References"
\DefineBibliographyStrings{english}{bibliography = {References}}

%Set the sort order of the names in each bibliography entry
\DeclareNameAlias{default}{last-first}

%Don't print the article title. To print the title, add #1 to the last {}
\DeclareFieldFormat[article,incollection,unpublished]{title}{}

%Add "vol." and "no." before volume and issue.
\DeclareFieldFormat[article]{volume}{\bibstring{volume}\addspace #1}
\DeclareFieldFormat[article]{number}{\bibstring{number}\addspace #1}

%Ensure that a comma follows abbreviated journal titles.
\DeclareFieldFormat{journaltitle}{\mkbibemph{#1}\isdot}

%Put a comma between the volume and issue instead of period.
\renewbibmacro*{volume+number+eid}{%
  \printfield{volume}%
  \setunit{\addcomma\space}%<---- was \setunit*{\adddot}%
  \printfield{number}%
  \setunit{\addcomma\space}%
  \printfield{eid}}

%Add a comma after the journal title.
\renewbibmacro*{journal+issuetitle}{%
  \usebibmacro{journal}%
  \setunit*{\addcomma\addspace}%<---- was \setunit*{\addspace}%
  \iffieldundef{series}
    {}
    {\newunit
     \printfield{series}%
     \setunit{\addspace}}%
  \usebibmacro{volume+number+eid}%
  \setunit{\addspace}%
  \usebibmacro{issue+date}%
  \setunit{\addcolon\space}%
  \usebibmacro{issue}%
  \newunit}

%Only print URL if doi is not present.
\DeclareFieldFormat{url}{%
  \iffieldundef{doi}{%
    \mkbibacro{URL}\addcolon\space\url{#1}%
  }{%
  }%
}
\DeclareFieldFormat{urldate}{%
  \iffieldundef{doi}{%
    \mkbibparens{\bibstring{urlseen}\space#1}%
  }{%
  }%
}

%Remove publisher from being printed.
\renewbibmacro*{publisher+location+date}{%
  \printlist{location}%
  \setunit*{\addcomma\space}%
  \usebibmacro{date}%
  \newunit}

%Fix in-text full citations
\DeclareCiteCommand{\fullcite}
  {\usebibmacro{prenote}}
  {\usedriver
     {\defcounter{minnames}{99}%
      \defcounter{maxnames}{99}}
     {\thefield{entrytype}}}
  {\multicitedelim}
  {\usebibmacro{postnote}}

%Use fancy tables.
\usepackage{booktabs}

%Set up todo notes in the PDF file
% \usepackage{todonotes}

%Use and set up the caption package for nicer captions.
\usepackage{caption}
\DeclareCaptionLabelFormat{bf}{\textbf{#1 #2}}
\captionsetup{
    font=small,
    labelsep=colon,
    labelformat=bf ,
    figurewithin=chapter,
    tablewithin=chapter,
}

\usepackage[subfigure]{tocloft}
\cftsetpnumwidth{1.55em}
\cftsetrmarg{4em plus 1fil}
\renewcommand{\cftchapfont}{\bfseries}
\renewcommand{\cftchappresnum}{Chapter }
\newlength{\mylen} % a "scratch" length
\settowidth{\mylen}{\bfseries\cftchappresnum\cftchapaftersnum} % extra space
\addtolength{\cftchapnumwidth}{\mylen} % add the extra space

% \usepackage{titlesec}
% \usepackage{titletoc}

% \titleformat{\chapter}[display]{\normalfont\Huge\bfseries}{Chapter \thechapter}{0.7em}{}
% \titleformat{\section}{\normalfont\LARGE\bfseries}{\thesection}{0.5em}{}
% \titleformat{\subsection}{\normalfont\Large\bfseries}{\thesubsection}{1em}{}
% \titleformat{\subsubsection}{\normalfont\large\bfseries}{\thesubsubsection}{1em}{}

% \titlecontents{chapter}[0pc]{}{\bfseries Chapter \thecontentslabel\quad}{}{\titlerule*[0.5pc]{.}\contentspage}
% \titlecontents{section}[1em]{}{\thecontentslabel\quad}{}{\titlerule*[0.5pc]{.}\contentspage}
% \titlecontents{subsection}[2em]{}{\thecontentslabel\quad}{}{\titlerule*[0.5pc]{.}\contentspage}
% \titlecontents{subsubsection}[3em]{}{\thecontentslabel\quad}{}{\titlerule*[0.5pc]{.}\contentspage}

\setcounter{secnumdepth}{3}
\setcounter{tocdepth}{3}

%Use the subfigure package
\usepackage{subfig}

%Allow table cells to span multiple rows.
\usepackage{multirow}

%Allow landscape rotated figures and captions.
\usepackage{afterpage}
\usepackage{rotating}
\usepackage{pdflscape}

%Set the root path where figures are stored.
\graphicspath{ {/home/bryan/dissertation/figures/} }

%Set a convenience command for table cells that allow line breaks.
\newcommand{\linebreakcell}[2][c]{%
    \begin{tabular}[#1]{@{}c@{}}#2\end{tabular}
}

%Add a dummy command for previously highlighted text
\newcommand*{\hl}[1]{#1}%

%Use and set up the siunitx package for nice units printing.
\usepackage{siunitx}
\sisetup{%
    group-separator = {,},
    range-phrase = {\text{ to }},
    list-separator = {\text{, }},
    list-final-separator = {\text{, and }},
    list-pair-separator = {\text{ and }},
}%
\DeclareSIUnit\calorie{cal}
\DeclareSIUnit\atmosphere{atm}
\DeclareSIUnit\torr{Torr}
\DeclareSIUnit\inch{in}

%Add the \ce command to easily specify chemicals
\usepackage[version=3]{mhchem}

%Declare convenience macros for printing the
%names of the alcohols.
\newcommand{\iPeOH}{\textit{i}-pentanol}
\newcommand{\nBuOH}{\textit{n}-butanol}
\newcommand{\sBuOH}{\textit{s}-butanol}
\newcommand{\tBuOH}{\textit{t}-butanol}
\newcommand{\iBuOH}{\textit{i}-butanol}

%The floatrow package allows multiple floats in a row
%and is set so that table captions are on top of the
%table.
\usepackage{floatrow}
\floatsetup[table]{style=plaintop}

%Use the titling package to allow easy access to custom title pages
\usepackage{titling}
\title{High Pressure Ignition Chemistry of Alternative Fuels}
\author{Bryan William Weber}

%Add bibliography and indices to the TOC
\usepackage{tocbibind}

%Improve handling of appendices
\usepackage{appendix}

%Use package that allows inline patching of commands. This is used in
%the appendices section.
\usepackage{xpatch}

%Use the bookmark package (which loads hyperref) so that only one
%compilation is necessary to get references.
\usepackage{bookmark}

%Set the color of the links and PDF metadata
\hypersetup{%
    pdfinfo={
        Title={High Pressure Ignition Chemistry of Alternative Fuels},
        Author={Bryan W. Weber},
    },
    colorlinks=true,
    citecolor=black,
    linkcolor=black,
    plainpages=false,
    final,
}

%Allow lualatex to properly add links processed from pax files.
\usepackage{pdftexcmds}
\makeatletter
\let\pdfescapename=\pdf@escapename
\let\pdfstrcmp=\pdf@strcmp
\makeatother
\usepackage{pax}

%Allow to use \doi to link to DOI links.
\usepackage{doi}

%Allow inserting PDF documents directly to the output. According to
%http://tex.stackexchange.com/a/13660/32374, should come after hyperref
\usepackage[final]{pdfpages}

%Do a better job with the automatic references. According to
%http://tex.stackexchange.com/a/1868/32374, should come after hyperref
\usepackage[capitalise, sort&compress]{cleveref}

%Set the auto-format names for the cleveref operations
\crefname{chapter}{Chapter}{Chapters}
\Crefname{chapter}{Chapter}{Chapters}
\crefname{section}{Sec.}{Secs.}
\Crefname{section}{Section}{Sections}
\crefname{subsection}{Sec.}{Secs.}
\Crefname{subsection}{Section}{Sections}
\crefname{subsubsection}{Sec.}{Secs.}
\Crefname{subsubsection}{Section}{Sections}
\crefname{figure}{Fig.}{Figs.}
\Crefname{figure}{Figure}{Figures}
\crefname{table}{Table}{Tables}
\Crefname{table}{Table}{Tables}
\crefname{equation}{Eq.}{Eqs.}
\Crefname{equation}{Equation}{Equations}
\crefname{appchap}{Appendix}{Appendices}
\Crefname{appchap}{Appendix}{Appendices}

\newcommand{\creflastconjunction}{, and~}
\newcommand{\crefrangeconjunction}{--}

%Set the size of the margins and the paper
%According to http://tex.stackexchange.com/a/26592/32374
%this should go after hyperref
\usepackage[
    margin=1in, 
    letterpaper, 
    head=15pt,
    % showframe,
]{geometry}

%Set up the page numbers
%This has to go after geometry so the page number is centered
\usepackage{fancyhdr}
\fancypagestyle{main}
{
    \fancyhf{}
    \fancyfoot[C]{\thepage}
    \renewcommand{\headrulewidth}{0pt}
}
\fancypagestyle{abstract}
{
   \fancyhf{}
   \fancyhead[C]{Bryan William Weber -- University of Connecticut, 2014}
   \renewcommand{\headrulewidth}{0pt}
}


\begin{document}

\begin{figure}[!ht]\CenterFloatBoxes
    \begin{floatrow}
        \killfloatstyle\ttabbox
        {\captionsetup{type=table}\caption{HHV of Ethanol, \iPeOH{}, and Gasoline}
        \label{tab:ipeoh-heats}}
        {\begin{tabular}{*{4}{c}}
            \toprule
            Compound & Ethanol \cite{Afeefy2014} & \iPeOH \cite{Afeefy2014} & Gasoline \cite{Davis2013} \\
            \midrule
            HHV [\si[per-mode=symbol]{\mega\joule\per\kilo\gram}] & 29.67 & 37.73 & 48.46 \\
            \bottomrule
        \end{tabular}}
        \ffigbox[\FBwidth]
            {\includegraphics[width=5cm]{04-Pentanol/ipeoh-skeletal}}
            {\caption{Skeletal structure of \iPeOH{}}
            \label{fig:ipeoh-skeletal}}
    \end{floatrow}
\end{figure}

\section{Structure of \textit{i}-Pentanol}
\label{sec:ipeoh-struct}

\textit{i}-Pentanol (3-methyl-1-butanol) is a five-carbon alcohol whose skeletal
structure is shown in \cref{fig:ipeoh-skeletal}. \textit{i}-Pentanol can
be produced biologically, and offers several similar advantages as the butanol
isomers as compared to ethanol. \Cref{tab:ipeoh-heats} compares the HHV
of ethanol, \iPeOH{}, and gasoline.

\section{Procedures}
\label{sec:ipeoh-procedure}

Experiments for \iPeOH{} in the RCM have been performed at the conditions listed
in \cref{tab:ipeoh-expts}. Homogeneous fuel and air pre-mixtures are prepared in an approximately
\SI{17}{\liter} mixing tank. The mixing tank and all tubes and manifolds
connecting the tank with the RCM are heated, allowing the study of
relatively low vapor pressure fuels. The initial temperature is set above
the saturation temperature of \iPeOH{} for each mixture studied. The mixing
tank is equipped with a magnetic stirrer to ensure complete homogeneity
of the mixture.

\begin{table}
    \caption{\iPeOH{} Experimental Conditions}
    \label{tab:ipeoh-expts}
    \begin{tabular}{*{5}{c}}
    \toprule
    \multicolumn{3}{c}{Reactant (Purity)} & \multirow{3}[0]{*}{\linebreakcell{Equivalence \\ Ratio \\ $\phi$}} & \multirow{3}[0]{*}{\linebreakcell{Compressed \\ Pressure \\ $P_C$ (bar)}} \\
    \cmidrule{1-3}
    \linebreakcell{\iPeOH{} \\ (99.6\%)} & \linebreakcell{O$_2$ \\ (99.994\%)} & \linebreakcell{N$_2$ \\ (99.999\%)} & & \\
    \cmidrule{1-3}
    \multicolumn{3}{c}{Mole Percentage}   & & \\
    \midrule
    2.41 & 20.50 & 77.09 & 1.0 & 7 \\
    2.41 & 20.50 & 77.09 & 1.0 & 20 \\
    2.41 & 20.50 & 77.09 & 1.0 & 40 \\
    1.22 & 20.75 & 78.03 & 0.5 & 7 \\
    1.22 & 20.75 & 78.03 & 0.5 & 20 \\
    1.22 & 20.75 & 78.03 & 0.5 & 40 \\
    4.71 & 20.01 & 75.27 & 2.0 & 20 \\
    4.71 & 20.01 & 75.27 & 2.0 & 40 \\
    \bottomrule
    \end{tabular}
\end{table}

Prior to mixture preparation, the mixing tank is vacuumed to less than
\SI{1}{\torr}, whereupon liquid fuel (\iPeOH{}, Sigma-Aldrich, 99.6\%
purity) is injected by a syringe through a septum. The syringe is massed
before and after the injection, with the difference being the amount of
fuel in the mixing tank. Based on this mass, required proportions of
the gaseous oxidizer (O2, 99.994\% purity, N2, 99.999\% purity) are
calculated. The gases are added to the mixing tank sequentially at
room temperature and the total pressure is monitored to ensure the
proper mixture concentrations are attained. Finally, the heaters and
stirring vane are switched on and the system is allowed approximately
\SI{1.5}{\hour} to reach steady state.

\section{Model Improvements}
Through collaboration with researchers at Lawrence Livermore National
Laboratory, many improvements to the chemical kinetic model for \iPeOH{} were
made relative to the work of \textcite{Tsujimura2012}. Some of the major
improvements are highlighted below; see the article for more detail
\cite{Sarathy2013}.

\begin{enumerate}
\item The model was restructured based on work with C$_4$ and C$_5$
      alcohols \cite{Sarathy2012, Heufer2012a}
\item The most stable conformers of \iPeOH{} were calculated using
      quantum chemistry software
\item The BDEs of the of the C-C, C-H, C-O, and O-H bonds were calculated
      using quantum chemistry software
\item The model includes the Waddington pathway shown to be important in
      low-temperature decomposition of \iPeOH{} by \textcite{Welz2012}
\item New reaction pathways were added based on the work of \textcite{Welz2012,
      Welz2013}, including the unconventional water-elimination pathway discussed
      in \textcite{Welz2013}
\end{enumerate}

Moreover, the following data sets from the literature and presented in
\cite{Sarathy2013} were used to validate the newly updated model, in
addition to the data presented here.

\begin{enumerate}
\item Ignition delays measured in a shock tube \cite{Tang2013, Tsujimura2012}
\item JSR species data \cite{Dayma2011}
\item New ignition delays measured in shock tubes \cite{Sarathy2013}
\item New JSR species data \cite{Sarathy2013}
\item New flame speed and flame extinction measurements \cite{Sarathy2013}
\end{enumerate}

\section{Experimental \& Modeling Results}
\label{sec:ipeoh-results}

The experimental ignition delays measured in the RCM are shown in
\cref{fig:ipeoh-7bar,fig:ipeoh-20bar,,fig:ipeoh-40bar}, along with
ignition delays measured in the shock tube and comparison with
the model simulations. There is no $\phi=2.0$ data set for \SI{7}{\atmosphere}
because no conditions at which ignition occurred could be found.
In \cref{fig:ipeoh-7bar,fig:ipeoh-20bar,fig:ipeoh-40bar}, solid
lines represent adiabatic, constant volume simulations, and dashed
lines represent volume-profile simulations.

\begin{figure}
\begin{floatrow}
    \ffigbox[\FBwidth]
        {
            \ffigbox[\FBwidth]
                {\includegraphics[width=6cm]{04-Pentanol/ipeoh-7bar}}
                {\caption{Shock tube and RCM ignition delay times from
                \textcite{Tsujimura2012} at \SI{7}{\atmosphere} compared
                with model predictions by the model from \textcite{Sarathy2013}.}
                \label{fig:ipeoh-7bar}}
            \par
            \ffigbox[\FBwidth]
                {\includegraphics[width=6cm]{04-Pentanol/ipeoh-20bar}}
                {\caption{Shock tube and RCM ignition delay times from
                \textcite{Tsujimura2012} at \SI{20}{\atmosphere} compared
                with model predictions by the model from \textcite{Sarathy2013}.}
                \label{fig:ipeoh-20bar}}
            \par
            \ffigbox[\FBwidth]
                {\includegraphics[width=6cm]{04-Pentanol/ipeoh-40bar}}
                {\caption{Shock tube and RCM ignition delay times from
                \textcite{Sarathy2013} at \SI{40}{\atmosphere} compared
                with model predictions by the model from \textcite{Sarathy2013}.}
                \label{fig:ipeoh-40bar}}
        }
        {\caption{Ignition delays of \iPeOH{}}\label{fig:ipeoh-expts}}
    \ffigbox[\FBwidth]
        {
            \ffigbox[\FBwidth]
                {\includegraphics[width=6cm]{04-Pentanol/ipeoh-7bar}}
                {\caption{Shock tube and RCM ignition delay times from
                \textcite{Tsujimura2012} at \SI{7}{\atmosphere} compared
                with model predictions by the model from \textcite{Sarathy2013}.}
                \label{fig:ipeoh-7bar}}
            \par
            \ffigbox[\FBwidth]
                {\includegraphics[width=6cm]{04-Pentanol/ipeoh-20bar}}
                {\caption{Shock tube and RCM ignition delay times from
                \textcite{Tsujimura2012} at \SI{20}{\atmosphere} compared
                with model predictions by the model from \textcite{Sarathy2013}.}
                \label{fig:ipeoh-20bar}}
            \par
            \ffigbox[\FBwidth]
                {\includegraphics[width=6cm]{04-Pentanol/ipeoh-40bar}}
                {\caption{Shock tube and RCM ignition delay times from
                \textcite{Sarathy2013} at \SI{40}{\atmosphere} compared
                with model predictions by the model from \textcite{Sarathy2013}.}
                \label{fig:ipeoh-40bar}}
        }
        {\caption{Ignition delays of \iPeOH{}}\label{fig:ipeoh-expts}}
\end{floatrow}
\end{figure}

At \SI{7}{\atmosphere}, the high-temperature ignition delays
are generally predicted to within a factor of 1.5. The RCM experiments
are also well predicted at low temperature---within a factor of 2---but
the disagreement grows to approximately a factor of 4 in the
intermediate temperature regime. At \SI{20}{\atmosphere}, the
high-temperature ignition delays are well predicted, including
capturing the equivalence ratio sensitivity of the ignition delays.
The ignition delays measured in the RCM are fairly well predicted
at the lean and stoichiometric condition, but are over-predicted
at the rich condition.
\end{document}
