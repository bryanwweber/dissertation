% arara: lualatex: { synctex: on, shell: off }
% arara: biber
% arara: lualatex: { synctex: on, shell: off }
% arara: sumatrapdf
\documentclass[../main.tex]{subfiles}
\begin{document}

The fast sampling system (FSS) used in this work is a commercial system supplied by
SME-Tec Gmbh. from Germany. The FSS is composed of two parts, the gas sampling valve
(GSV) and the Controller. A schematic of the GSV is shown in \cref{fig:valve-schematic}.
Gases are admitted from the reaction chamber into the heated carrying tubes
through the poppet-style valve on the left of the GSV. The sampled gases are then
conducted through the GSV outlet into the \SI{15}{\milli\liter} sampling bottle.

A schematic of the GSV assembly is shown in \cref{fig:gsv-assem-schematic}.
The GSV is mounted to the RCM by a custom-made end plug. The reaction chamber
is sealed by an O-ring on the small- and large-diameter portions of the GSV.
The depth that the GSV is inserted into the reaction chamber is adjustable
by adding or removing shims in the end plug assembly. The insertion depth
is chosen so that the tip of the GSV is outside the boundary layer on the
end wall.

The portion of the GSV protruding into the reaction chamber has minimal
effect on the homogeneity of the reaction chamber. Moreover, the removal of
samples has minimal effect on the measured ignition delay.
This has been verified experimentally by measuring the ignition delay with
and without the GSV present, and with and without sampling occurring.
In both cases, the difference in ignition delay was statistically insignificant
for $\alpha=0.05$.

The close-open-close (COC) cycle of the GSV is controlled by a mass-spring
system. The poppet face is connected to a rod running the length of the
GSV and connected to the mass at the rear of the valve. To open the poppet,
the mass is accelerated forward by passing current through the coil around
the mass. The rod is also
connected to a spring that is used to restore the poppet to its original
position after being extended.

The GSV has an adjustable COC time, by adjusting the
distance the plate is allowed to move. Furthermore, the GSV has the ability
to measure the displacement of the mass, allowing the direct measurement of
the COC time and the absolute time of opening.

The GSV controller is triggered by a \SI{5}{\volt} signal from the cDAQ.
The timing of the trigger signal is controlled by the LabView VI. The pressure
signal from the reaction chamber is read from the cDAQ in \SI{1}{\milli\second}
chunks in a loop. On each loop iteration, the maximum pressure is checked
against a desired trigger pressure; when the reaction chamber pressure exceeds
the trigger pressure, the cDAQ sends the trigger to the GSV controller. The
GSV controller has an adjustable delay (\SIrange{4.5}{70}{\milli\second}) that
is used to control the timing of the opening of the GSV during the induction period.
The absolute opening time of the GSV is thus dependent on three parameters:
\begin{enumerate}
\item The cable delays from the PC to the cDAQ; from the cDAQ to the GSV
      controller; and from the GSV controller to the GSV itself
\item The processing time of the LabView VI
\item The delay set in the GSV controller.
\end{enumerate}

However, since the absolute opening time of the GSV can be measured by
the signal sent from the GSV to the controller (and thence to the cDAQ),
the uncertainty in the opening time is actually quite small, and is related
to the cable delay in sending the COC signal from the GSV to the cDAQ and
the precision of the A/D conversion in the cDAQ. The uncertainty of the
absolute opening time is thus estimated as $\pm \SI{1}{\micro\second}$. \todo{Verify this number}

Furthermore, the correspondence of the COC signal from the valve with
the physical valve movements has been experimentally verified by
high speed video. Frames from these videos are shown in \cref{fig:valve-video}.
The frames show that the opening and closing times of the valve face
correspond closely with repeatable points in the peak of the voltage.

\missingfigure{GSV Video}

\end{document}
