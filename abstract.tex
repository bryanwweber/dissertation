% arara: lualatex
% \documentclass{article}
% \usepackage{xcolor}
% \usepackage{soul}
% \usepackage{siunitx}
% \newcommand{\iPeOH}{\textit{i}-pentanol}
% \newcommand{\nBuOH}{\textit{n}-butanol}
% \newcommand{\sBuOH}{\textit{s}-butanol}
% \newcommand{\tBuOH}{\textit{t}-butanol}
% \newcommand{\iBuOH}{\textit{i}-butanol}

% \begin{document}
Using the Rapid Compression Machine at the University of Connecticut,
studies of the autoignition behavior of alternative fuels are conducted
in the temperature range \SIrange{650}{900}{\kelvin} and the pressure
range \SIrange{15}{50}{\bar} \hl{with a focus
on developing a fundamental understanding of the chemistry controlling
the autoignition of alternative fuels at engine relevant---high-pressure and low-to-intermediate
temperature---conditions. The alternative fuels
studied here include five bio-alcohol fuels---\nBuOH{}, \sBuOH{}, \tBuOH{},
\iBuOH{}, and \iPeOH{}---that are studied to investigate the effect of
the alcohol group and molecular structures on autoignition behavior. In
addition, methylcyclohexane---an important component of fuels derived from
alternative petroleum sources and a component in surrogate fuel
formulations---is studied.}

The ignition delay of the \hl{alcohols} shows no evidence of phenomena such
as two-stage ignition and negative temperature coefficient (NTC) of the
ignition delay. However, the relative reactivity shows a complicated
dependence on the molecular structure and the pressure and temperature
conditions. \hl{Moreover, \iPeOH{} and \tBuOH{} show similar heat
release behavior prior to the main ignition event.} These results are
explained \hl{through detailed chemical kinetic analysis} in terms of the
unique chemistry possible in each \hl{alcohol} because
of their unique structures.

The ignition behavior of methylcyclohexane is similar to alkanes
and other cycloalkanes, in that methylcyclohexane shows strong two-stage
ignition and NTC behavior. \hl{Analysis of a detailed kinetic model shows
that t}he prominent reaction pathways causing this behavior are also
similar between methylcyclohexane and similar molecules, indicating that
the reaction types controlling the autoignition behavior of hydrocarbons
are common among many fuel structures. In addition, gas samples extracted
from the reaction chamber during the induction period are used to identify
and quantify important intermediate species during the autoignition of
methylcyclohexane.

The experimental data developed in this work provide a comprehensive set
of archival ignition data that can be used to benchmark and validate
chemical kinetic models for the combustion of alternative fuels. These
data also indicate the remaining gaps in the understanding of the
high-pressure ignition chemistry of alternative fuels and provide preliminary
directions for future research to close the gaps.
% \end{document}
