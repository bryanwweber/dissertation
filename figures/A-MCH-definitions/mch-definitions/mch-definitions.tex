\documentclass{article}
\usepackage[letterpaper, hmargin=1in, vmargin=0.875in, showframe]{geometry}
\setlength{\parindent}{0pt}
\pagestyle{empty}


% Copy of relevant parts from the main preamble.
\usepackage{mathtools}

% Uncomment the following to use Times New Roman and Cambria Math
% \usepackage{unicode-math}
% \unimathsetup{math-style=TeX}
% \setmathfont[range=\mathup/{num}]{Times New Roman}
% \setmathfont[range=\mathit/{greek,Greek,latin,Latin}]{Cambria Math}
% \setmathfont[range=\mathup/{greek,Greek,latin,Latin}]{Cambria Math}
% \setmathfont[range={"2212,"002B,"003D,"0028,"0029,"005B,"005D,"221A,
% "2211,"2248,"222B,"007C,"2026,"2202,"00D7,"0302,"2261,"0025,"22C5,
% "00B1,"2194,"21D4,"2032}]
% {Cambria Math}
% \setmainfont[Ligatures=TeX]{Times New Roman}

% Uncomment the following to use Linux Libertine
% \usepackage[libertine]{newtxmath}
% \usepackage[no-math]{fontspec}
% \setmainfont{Linux Libertine O}

% Uncomment the following to use TeX Gyre Termes
% \usepackage{unicode-math}
% \unimathsetup{math-style=TeX}
% \setmainfont{TeX Gyre Termes}
% \setmathfont{TeX Gyre Termes Math}

% Uncomment the following to use TeX Gyre Pagella
\usepackage{unicode-math}
\unimathsetup{math-style=TeX}
\setmainfont{TeX Gyre Pagella}
\setmathfont{TeX Gyre Pagella Math}

% \usepackage[sfmath]{kpfonts}
% \renewcommand*\familydefault{\sfdefault}
% \usepackage[T1]{fontenc}

% \usepackage[no-math]{fontspec}
% Always use Inconsolata
\setmonofont{Inconsolata}

\usepackage{microtype}


\usepackage[version=3]{mhchem}
\usepackage{chemfig}
\definesubmol\nobond{-[,0.2,,,draw=none]}
\setatomsep{2.25em}
\usetikzlibrary{matrix}
\tikzset{line join=bevel}

\begin{document}
\begin{center}%
    \begin{tikzpicture}%
        \matrix[%
            matrix of nodes,%
            column sep=-\pgflinewidth, row sep=-\pgflinewidth,%
            nodes={rectangle, draw, minimum width=2.1in, anchor=north},%
            row 1/.append style={nodes={minimum height=1.75in}},%
            row 2/.append style={nodes={minimum height=1.75in}},%
            row 3/.append style={nodes={minimum height=2in}},%
            row 4/.append style={nodes={minimum height=2in}},%
            row 5/.append style={nodes={minimum height=2in}},%
            row 6/.append style={nodes={minimum height=2in}},%
            row 7/.append style={nodes={minimum height=2in}},%
            nodes in empty cells,
            anchor=north,
        ] (chems) %
        {%
        \chemname{\chemfig{CH_3-[6]*6(------)}}{mch} &%
        \chemname{\chemfig{CH_3-[6]\lewis{0.,C}*6(------)}}{mchr1} &%
        \chemname{\chemfig{CH_3-[6]*6(-----\lewis{0.,CH}-)}}{mchr2} \\%
        \chemname{\chemfig{CH_3-[6]*6(----\lewis{0.,CH}--)}}{mchr3} &%
        \chemname{\chemfig{CH_3-[6]*6(---\lewis{0.,CH}---)}}{mchr4} &%
        \chemname{\chemfig{{C}|\lewis{0.,H_2}-[6,,1]*6(------)}}{cychexch2} \\%
        & & \\
        };%
        \draw (chems-2-1.base west) -- (chems-2-3.base east);
    \end{tikzpicture}%
\end{center}%
\end{document}