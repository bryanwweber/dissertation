
% Copy of relevant parts from the main preamble.
\usepackage{mathtools}

% Uncomment the following to use Times New Roman and Cambria Math
% \usepackage{unicode-math}
% \unimathsetup{math-style=TeX}
% \setmathfont[range=\mathup/{num}]{Times New Roman}
% \setmathfont[range=\mathit/{greek,Greek,latin,Latin}]{Cambria Math}
% \setmathfont[range=\mathup/{greek,Greek,latin,Latin}]{Cambria Math}
% \setmathfont[range={"2212,"002B,"003D,"0028,"0029,"005B,"005D,"221A,
% "2211,"2248,"222B,"007C,"2026,"2202,"00D7,"0302,"2261,"0025,"22C5,
% "00B1,"2194,"21D4,"2032}]
% {Cambria Math}
% \setmainfont[Ligatures=TeX]{Times New Roman}

% Uncomment the following to use Linux Libertine
% \usepackage[libertine]{newtxmath}
% \usepackage[no-math]{fontspec}
% \setmainfont{Linux Libertine O}

% Uncomment the following to use TeX Gyre Termes
% \usepackage{unicode-math}
% \unimathsetup{math-style=TeX}
% \setmainfont{TeX Gyre Termes}
% \setmathfont{TeX Gyre Termes Math}

% Uncomment the following to use TeX Gyre Pagella
\usepackage{unicode-math}
\unimathsetup{math-style=TeX}
\setmainfont{TeX Gyre Pagella}
\setmathfont{TeX Gyre Pagella Math}

% \usepackage[sfmath]{kpfonts}
% \renewcommand*\familydefault{\sfdefault}
% \usepackage[T1]{fontenc}

% \usepackage[no-math]{fontspec}
% Always use Inconsolata
\setmonofont{Inconsolata}

\usepackage{microtype}
