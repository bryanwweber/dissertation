\documentclass{standalone}


% Copy of relevant parts from the main preamble.
\usepackage{mathtools}

% Uncomment the following to use Times New Roman and Cambria Math
% \usepackage{unicode-math}
% \unimathsetup{math-style=TeX}
% \setmathfont[range=\mathup/{num}]{Times New Roman}
% \setmathfont[range=\mathit/{greek,Greek,latin,Latin}]{Cambria Math}
% \setmathfont[range=\mathup/{greek,Greek,latin,Latin}]{Cambria Math}
% \setmathfont[range={"2212,"002B,"003D,"0028,"0029,"005B,"005D,"221A,
% "2211,"2248,"222B,"007C,"2026,"2202,"00D7,"0302,"2261,"0025,"22C5,
% "00B1,"2194,"21D4,"2032}]
% {Cambria Math}
% \setmainfont[Ligatures=TeX]{Times New Roman}

% Uncomment the following to use Linux Libertine
% \usepackage[libertine]{newtxmath}
% \usepackage[no-math]{fontspec}
% \setmainfont{Linux Libertine O}

% Uncomment the following to use TeX Gyre Termes
% \usepackage{unicode-math}
% \unimathsetup{math-style=TeX}
% \setmainfont{TeX Gyre Termes}
% \setmathfont{TeX Gyre Termes Math}

% Uncomment the following to use TeX Gyre Pagella
\usepackage{unicode-math}
\unimathsetup{math-style=TeX}
\setmainfont{TeX Gyre Pagella}
\setmathfont{TeX Gyre Pagella Math}

% \usepackage[sfmath]{kpfonts}
% \renewcommand*\familydefault{\sfdefault}
% \usepackage[T1]{fontenc}

% \usepackage[no-math]{fontspec}
% Always use Inconsolata
\setmonofont{Inconsolata}

\usepackage{microtype}


\usepackage[version=3]{mhchem}
\usepackage{chemfig}
\definesubmol\nobond{-[,0.2,,,draw=none]}
\setatomsep{2.25em}
\usetikzlibrary{positioning, calc, arrows.meta}
\tikzset{
    flux/.style={
    right,
    align=left,
    },
}
\newcommand*{\flux}[3]{\SI{#1}{\percent}\\\textbf{\SI{#2}{\percent}}\\\textit{\SI{#3}{\percent}}}
\usepackage{siunitx}
\sisetup{detect-all=true}

\begin{document}
    \begin{tikzpicture}[x=1cm, y=1cm]
        \node (mch) {\chemfig{CH_3-[6]*6(------)}};
        \node[below right=0 and 0 of mch] (cychexch2) {\chemfig{{C}|\lewis{0.,H_2}-[6,,1]*6(------)}};
        \node[below left=-1 and 0 of cychexch2] (cychexch2oo) {\chemfig{\lewis{0.,O}-[6]O-[7]-[6]*6(------)}};
        \node[below right=0 and 3 of mch] (mchr3) {\chemfig{CH_3-[6]*6(--\lewis{0.,CH}----)}};
        \node[below right=0 and 0 of mchr3] (mchr3oo) {\chemfig{CH_3-[6]*6(----(-O-[1]\lewis{0.,O})--)}};
        \node[right=of mchr3] (mchr3s) {\chemfig{H_2C=^-[:60]-(-[:60]CH_3)-[:300]-[:240]\lewis{0.,CH_2}}};
        \node[below left=0 and 3 of mch] (mchr2) {\chemfig{CH_3-[6]*6(-----\lewis{0.,CH}-)}};
        \node[above left=-1 and 1 of mchr2, align=center] (cychexene) {\chemfig{\lewis{0.,CH_3}}\\[0.5\baselineskip]$+$\\[0.5\baselineskip]\chemfig{*6(---=--)}};
        \node[below left=of mchr2] (mchr2oo) {\chemfig{CH_3-[6]*6(-----(-O-[7]\lewis{0.,O})-)}};
        \node[right=of mchr2] (mchr2s) {\chemfig{H_2C=-[:300](-CH_3)-[:240]-[4]-[:240]\lewis{0.,CH_2}}};
        \node[above left=0 and 3 of mch] (mchr1) {\chemfig{CH_3-[6]\lewis{0.,C}*6(------)}};
        \node[left=of mchr1] (mchr1s) {\chemfig{[:15]H_3C-[::300](=[::60]CH_2)-[::-60]-[::60]-[::60]-[::-60]\lewis{0.,CH_2}}};
        \node[above left=of mchr1] (mchr1oo) {\chemfig{*6(----(-[:120]H_3C)(-[:60]O-[0]\lewis{0.,O})--)}};
        \node[above=2 of mch] (mchnumbers) {\chemfig[][scale=1.3]{CH_3-[6](-[6,0.3,,,draw=none]\color{red}1)*6(-(!\nobond \color{red}6)-(!\nobond \color{red}5)-(!\nobond \color{red}4)-(!\nobond \color{red}3)-(!\nobond \color{red}2)-)}};
        \node[above right=0 and 3 of mch] (mchr4) {\chemfig{CH_3-[6]*6(---\lewis{0.,CH}---)}};
        \node[right=of mchr4] (mchr4s) {\chemfig{[:-15]\lewis{0.,H_2C}-(-[::-60]CH_3)-[::60]-[::60]-[::60]=[::-60]H_2C}};
        \node[above right=of mchr4] (mchr4oo) {\chemfig{CH_3-[6]*6(---(-O-[:330]\lewis{0.,O})---)}};
    \end{tikzpicture}
\end{document}
