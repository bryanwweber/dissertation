%Set the package to import preambles
\usepackage{subfiles}

%Load graphicx here to specify options
\usepackage[final]{graphicx}

%Various math improvements.
%Must be loaded before hyperref
\usepackage{mathtools}

%Set the document font
%Set the math font. Has to come after mathtools because
%some font stuff gets overwritten.

% Uncomment the following lines to use Times New Roman for text and numbers
% and Cambria Math for math fonts
% \usepackage{unicode-math}
% \unimathsetup{math-style=TeX}
% \setmathfont[range=\mathup/{num}]{Times New Roman}
% \setmathfont[range=\mathit/{greek,Greek,latin,Latin}]{Cambria Math}
% \setmathfont[range=\mathup/{greek,Greek,latin,Latin}]{Cambria Math}
% \setmathfont[range={"2212,"002B,"003D,"0028,"0029,"005B,"005D,"221A,
% "2211,"2248,"222B,"007C,"2026,"2202,"00D7,"0302,"2261,"0025,"22C5,
% "00B1,"2194,"21D4,"2260}]
% {Cambria Math}
% \setmainfont[Ligatures=TeX]{Times New Roman}

% Uncomment the following lines to use TeX Gyre Termes for text and math
% fonts. This is very similar to Times and was used for the "official"
% versions of the thesis.
% \usepackage{unicode-math}
% \unimathsetup{math-style=TeX}
% \setmainfont{TeX Gyre Termes}
% \setmathfont{TeX Gyre Termes Math}

% Uncomment the following lines to use TeX Gyre Pagella for text and math
% fonts. This is the font that was used for printing my personal bound
% copies of the thesis.
% \usepackage{unicode-math}
% \unimathsetup{math-style=TeX}
% \setmainfont{TeX Gyre Pagella}
% \setmathfont{TeX Gyre Pagella Math}

% Uncomment the following lines to use Linux Libertine for text and math
% fonts.
% \usepackage[libertine]{newtxmath}
% \usepackage[no-math]{fontspec}
% \setmainfont{Linux Libertine O}

% Uncomment the following lines to use Charis SIL for the text and math
% fonts.
% \usepackage[charter]{mathdesign}  % sets the math font
% \usepackage[no-math]{fontspec}
% \setmainfont[
    % Extension = .ttf,
    % BoldFont = CharisSIL-B,
    % BoldItalicFont = CharisSIL-BI,
    % ItalicFont = CharisSIL-I,
% ]{CharisSIL-R}          % sets the document font

% Uncomment the following for Iwona sans-serif fonts
% \usepackage[math]{iwona}
% \usepackage[no-math]{fontspec}
% \setmainfont{Iwona}

\usepackage[sfmath]{kpfonts}
\renewcommand*\familydefault{\sfdefault}
\usepackage[T1]{fontenc}

\usepackage[no-math]{fontspec}
% Always use Inconsolata for the monospaced font.
\setmonofont{Inconsolata}

%Better looking fonts
\usepackage[final]{microtype}

% Load the CharisSIL options for microtype.
% \LoadMicrotypeFile{CharisSIL}

%Set the text to double spacing
%According to hyperref README,
%setspace should be loaded first
\usepackage[doublespacing]{setspace}

%Set a command to easily skip a line
\newcommand{\blankline}{\vspace*{\baselineskip}}

%Set up biblatex
\usepackage[
    backend=biber,
    % url=false,
    doi=true,
    sorting=none,
    sortcites=true,
    maxbibnames=6,
    minbibnames=6,
    maxcitenames=2,
    mincitenames=1,
    citestyle=numeric-comp,
    firstinits=true,
    isbn=false
]{biblatex}
\addbibresource{/home/bryan/dissertation/library.bib}

%Remove the "In:" from before the journal title for articles
\renewbibmacro{in:}{%
  \ifentrytype{article}{}{\printtext{\bibstring{in}\intitlepunct}}}

%Change the name of the bibliography section to "References"
\DefineBibliographyStrings{english}{bibliography = {References}}

%Set the sort order of the names in each bibliography entry
\DeclareNameAlias{default}{last-first}

%Don't print the article title. To print the title, add #1 to the last {}
\DeclareFieldFormat[article,incollection,unpublished]{title}{}

%Add "vol." and "no." before volume and issue.
\DeclareFieldFormat[article]{volume}{\bibstring{volume}\addspace #1}
\DeclareFieldFormat[article]{number}{\bibstring{number}\addspace #1}

%Ensure that a comma follows abbreviated journal titles.
\DeclareFieldFormat{journaltitle}{\mkbibemph{#1}\isdot}

%Put a comma between the volume and issue instead of period.
\renewbibmacro*{volume+number+eid}{%
  \printfield{volume}%
  \setunit{\addcomma\space}%<---- was \setunit*{\adddot}%
  \printfield{number}%
  \setunit{\addcomma\space}%
  \printfield{eid}}

%Add a comma after the journal title.
\renewbibmacro*{journal+issuetitle}{%
  \usebibmacro{journal}%
  \setunit*{\addcomma\addspace}%<---- was \setunit*{\addspace}%
  \iffieldundef{series}
    {}
    {\newunit
     \printfield{series}%
     \setunit{\addspace}}%
  \usebibmacro{volume+number+eid}%
  \setunit{\addspace}%
  \usebibmacro{issue+date}%
  \setunit{\addcolon\space}%
  \usebibmacro{issue}%
  \newunit}

%Only print URL if doi is not present.
\DeclareFieldFormat{url}{%
  \iffieldundef{doi}{%
    \mkbibacro{URL}\addcolon\space\url{#1}%
  }{%
  }%
}
\DeclareFieldFormat{urldate}{%
  \iffieldundef{doi}{%
    \mkbibparens{\bibstring{urlseen}\space#1}%
  }{%
  }%
}

%Remove publisher from being printed.
\renewbibmacro*{publisher+location+date}{%
  \printlist{location}%
  \setunit*{\addcomma\space}%
  \usebibmacro{date}%
  \newunit}

%Fix in-text full citations
\DeclareCiteCommand{\fullcite}
  {\usebibmacro{prenote}}
  {\usedriver
     {\defcounter{minnames}{99}%
      \defcounter{maxnames}{99}}
     {\thefield{entrytype}}}
  {\multicitedelim}
  {\usebibmacro{postnote}}

%Use fancy tables.
\usepackage{booktabs}

%Set up todo notes in the PDF file
% \usepackage{todonotes}

%Use and set up the caption package for nicer captions.
\usepackage{caption}
\DeclareCaptionLabelFormat{bf}{\textbf{#1 #2}}
\captionsetup{
    font=small,
    labelsep=colon,
    labelformat=bf ,
    figurewithin=chapter,
    tablewithin=chapter,
}

\usepackage[subfigure]{tocloft}
\cftsetpnumwidth{1.55em}
\cftsetrmarg{4em plus 1fil}
\renewcommand{\cftchapfont}{\bfseries}
\renewcommand{\cftchappresnum}{Chapter }
\newlength{\mylen} % a "scratch" length
\settowidth{\mylen}{\bfseries\cftchappresnum\cftchapaftersnum} % extra space
\addtolength{\cftchapnumwidth}{\mylen} % add the extra space

% \usepackage{titlesec}
% \usepackage{titletoc}

% \titleformat{\chapter}[display]{\normalfont\Huge\bfseries}{Chapter \thechapter}{0.7em}{}
% \titleformat{\section}{\normalfont\LARGE\bfseries}{\thesection}{0.5em}{}
% \titleformat{\subsection}{\normalfont\Large\bfseries}{\thesubsection}{1em}{}
% \titleformat{\subsubsection}{\normalfont\large\bfseries}{\thesubsubsection}{1em}{}

% \titlecontents{chapter}[0pc]{}{\bfseries Chapter \thecontentslabel\quad}{}{\titlerule*[0.5pc]{.}\contentspage}
% \titlecontents{section}[1em]{}{\thecontentslabel\quad}{}{\titlerule*[0.5pc]{.}\contentspage}
% \titlecontents{subsection}[2em]{}{\thecontentslabel\quad}{}{\titlerule*[0.5pc]{.}\contentspage}
% \titlecontents{subsubsection}[3em]{}{\thecontentslabel\quad}{}{\titlerule*[0.5pc]{.}\contentspage}

\setcounter{secnumdepth}{3}
\setcounter{tocdepth}{3}

%Use the subfigure package
\usepackage{subfig}

%Allow table cells to span multiple rows.
\usepackage{multirow}

%Allow landscape rotated figures and captions.
\usepackage{afterpage}
\usepackage{rotating}
\usepackage{pdflscape}

%Set the root path where figures are stored.
\graphicspath{ {/home/bryan/dissertation/figures/} }

%Set a convenience command for table cells that allow line breaks.
\newcommand{\linebreakcell}[2][c]{%
    \begin{tabular}[#1]{@{}c@{}}#2\end{tabular}
}

%Add a dummy command for previously highlighted text
\newcommand*{\hl}[1]{#1}%

%Use and set up the siunitx package for nice units printing.
\usepackage{siunitx}
\sisetup{%
    group-separator = {,},
    range-phrase = {\text{ to }},
    list-separator = {\text{, }},
    list-final-separator = {\text{, and }},
    list-pair-separator = {\text{ and }},
}%
\DeclareSIUnit\calorie{cal}
\DeclareSIUnit\atmosphere{atm}
\DeclareSIUnit\torr{Torr}
\DeclareSIUnit\inch{in}

%Add the \ce command to easily specify chemicals
\usepackage[version=3]{mhchem}

%Declare convenience macros for printing the
%names of the alcohols.
\newcommand{\iPeOH}{\textit{i}-pentanol}
\newcommand{\nBuOH}{\textit{n}-butanol}
\newcommand{\sBuOH}{\textit{s}-butanol}
\newcommand{\tBuOH}{\textit{t}-butanol}
\newcommand{\iBuOH}{\textit{i}-butanol}

%The floatrow package allows multiple floats in a row
%and is set so that table captions are on top of the
%table.
\usepackage{floatrow}
\floatsetup[table]{style=plaintop}

%Use the titling package to allow easy access to custom title pages
\usepackage{titling}
\title{High Pressure Ignition Chemistry of Alternative Fuels}
\author{Bryan William Weber}

%Add bibliography and indices to the TOC
\usepackage{tocbibind}

%Improve handling of appendices
\usepackage{appendix}

%Use package that allows inline patching of commands. This is used in
%the appendices section.
\usepackage{xpatch}

%Use the bookmark package (which loads hyperref) so that only one
%compilation is necessary to get references.
\usepackage{bookmark}

%Set the color of the links and PDF metadata
\hypersetup{%
    pdfinfo={
        Title={High Pressure Ignition Chemistry of Alternative Fuels},
        Author={Bryan W. Weber},
    },
    colorlinks=true,
    citecolor=black,
    linkcolor=black,
    plainpages=false,
    final,
}

%Allow lualatex to properly add links processed from pax files.
\usepackage{pdftexcmds}
\makeatletter
\let\pdfescapename=\pdf@escapename
\let\pdfstrcmp=\pdf@strcmp
\makeatother
\usepackage{pax}

%Allow to use \doi to link to DOI links.
\usepackage{doi}

%Allow inserting PDF documents directly to the output. According to
%http://tex.stackexchange.com/a/13660/32374, should come after hyperref
\usepackage[final]{pdfpages}

%Do a better job with the automatic references. According to
%http://tex.stackexchange.com/a/1868/32374, should come after hyperref
\usepackage[capitalise, sort&compress]{cleveref}

%Set the auto-format names for the cleveref operations
\crefname{chapter}{Chapter}{Chapters}
\Crefname{chapter}{Chapter}{Chapters}
\crefname{section}{Sec.}{Secs.}
\Crefname{section}{Section}{Sections}
\crefname{subsection}{Sec.}{Secs.}
\Crefname{subsection}{Section}{Sections}
\crefname{subsubsection}{Sec.}{Secs.}
\Crefname{subsubsection}{Section}{Sections}
\crefname{figure}{Fig.}{Figs.}
\Crefname{figure}{Figure}{Figures}
\crefname{table}{Table}{Tables}
\Crefname{table}{Table}{Tables}
\crefname{equation}{Eq.}{Eqs.}
\Crefname{equation}{Equation}{Equations}
\crefname{appchap}{Appendix}{Appendices}
\Crefname{appchap}{Appendix}{Appendices}

\newcommand{\creflastconjunction}{, and~}
\newcommand{\crefrangeconjunction}{--}

%Set the size of the margins and the paper
%According to http://tex.stackexchange.com/a/26592/32374
%this should go after hyperref
\usepackage[
    margin=1in, 
    letterpaper, 
    head=15pt,
    % showframe,
]{geometry}

%Set up the page numbers
%This has to go after geometry so the page number is centered
\usepackage{fancyhdr}
\fancypagestyle{main}
{
    \fancyhf{}
    \fancyfoot[C]{\thepage}
    \renewcommand{\headrulewidth}{0pt}
}
\fancypagestyle{abstract}
{
   \fancyhf{}
   \fancyhead[C]{Bryan William Weber -- University of Connecticut, 2014}
   \renewcommand{\headrulewidth}{0pt}
}
